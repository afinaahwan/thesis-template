%% LyX 2.1.2 created this file.  For more info, see http://www.lyx.org/.
%% Do not edit unless you really know what you are doing.
\documentclass[12pt,oneside,bahasa,american]{book}
\usepackage{charter}
\usepackage[latin9]{inputenc}
\usepackage[a4paper]{geometry}
\geometry{verbose,tmargin=3cm,bmargin=3cm,lmargin=4cm,rmargin=3cm}
\usepackage{fancyhdr}
\pagestyle{fancy}
\setcounter{secnumdepth}{3}
\setcounter{tocdepth}{3}
\usepackage{color}
\usepackage{babel}
\usepackage{array}
\usepackage{multirow}
\usepackage{graphicx}
\usepackage{setspace}
\usepackage[authoryear]{natbib}
\onehalfspacing
\usepackage[unicode=true,pdfusetitle,
 bookmarks=true,bookmarksnumbered=false,bookmarksopen=false,
 breaklinks=false,pdfborder={0 0 1},backref=false,colorlinks=false]
 {hyperref}

\makeatletter

%%%%%%%%%%%%%%%%%%%%%%%%%%%%%% LyX specific LaTeX commands.
\DeclareRobustCommand{\cyrtext}{%
  \fontencoding{T2A}\selectfont\def\encodingdefault{T2A}}
\DeclareRobustCommand{\textcyr}[1]{\leavevmode{\cyrtext #1}}
\AtBeginDocument{\DeclareFontEncoding{T2A}{}{}}

%% Because html converters don't know tabularnewline
\providecommand{\tabularnewline}{\\}

%%%%%%%%%%%%%%%%%%%%%%%%%%%%%% Textclass specific LaTeX commands.
\newenvironment{lyxcode}
{\par\begin{list}{}{
\setlength{\rightmargin}{\leftmargin}
\setlength{\listparindent}{0pt}% needed for AMS classes
\raggedright
\setlength{\itemsep}{0pt}
\setlength{\parsep}{0pt}
\normalfont\ttfamily}%
 \item[]}
{\end{list}}

%%%%%%%%%%%%%%%%%%%%%%%%%%%%%% User specified LaTeX commands.
%% \renewcommand{\chapter}[1]{\chapter[#1]{\centering #1}}
\usepackage{titlesec}
\usepackage{graphicx}
\usepackage{pdflscape}
\usepackage{colortbl}
\usepackage{amsmath}
\usepackage{color}
\usepackage{tocbibind}
\usepackage{url}
\usepackage[titletoc]{appendix}

\definecolor{lineno}{rgb}{0.5,0.5,0.5}
\definecolor{bcolor}{rgb}{0.98,0.98,0.98}
\definecolor{code}{rgb}{0,0.1,0.6}

\definecolor{codesty}{rgb}{0.7,0.1,0.1}
\definecolor{commandsty}{rgb}{0.1,0.5,0.0}
\definecolor{keywordsty}{rgb}{0.4,0.0,0.4}
\definecolor{stringsty}{rgb}{1.0,0.5,0.0}

\titleformat{\chapter}[display]
  {\normalfont\huge\bfseries\centering}{\chaptertitlename\ \thechapter}{20pt}{\Huge}


\renewcommand{\contentsname}{Table of Contents}

\newcommand{\Judul}{}
\newcommand{\JudulInggris}{Plagiarism Detection of Java Source Code}
\newcommand{\Name}{Ika Pretty Siregar}
\newcommand{\NPM}{53411487}
\newcommand{\NIRM}{}
\newcommand{\JenisTulisan}{Undergraduate Thesis}
%\newcommand{\Gelar}{Jenjang DIII / Setara Sarjana Muda}
\newcommand{\Fakultas}{Faculty of Industrial Technology}
\newcommand{\Department}{Informatics Engineering}
%\newcommand{\Prodi}{Direktorat Program Diploma Tiga Teknologi Informasi}
\newcommand{\Tahun}{2015}
\newcommand{\Bulan}{February}
\newcommand{\Tanggal}{24}
\newcommand{\Kota}{Jakarta}
\newcommand{\KataKunci}{Java, Plagiat Kode Sumber,  Levenshtein Distance, Deteksi Java }
\newcommand{\KeyWords}{Java, Source Code Plagiarism, Levenshtein Distance, Detection Java}
\newcommand{\KoordinatorPI}{}
\newcommand{\KetuaJurusan}{Haryanto, SSi, MM}
\newcommand{\Supervisor}{Dr. Lintang Yuniar Banowosari, S.Kom., M.Sc.}
\newcommand{\AnggotaPembimbingA}{Dr. rer. nat. I Made Wiryana, SSi., SKom., MAppSc}
\newcommand{\Supervisors}{Dr. Tubagus Maulana Kusuma, SKom., MEngSc.}
\newcommand{\AnggotaPembimbingB}{Dr. Lintang Yuniar Bonowosari, S.Kom., MSc.}
\newcommand{\KetuaUjian}{Dr. Ravi Ahmad Salim}
\newcommand{\SekUjian}{Prof. Dr. Wahyudi Priyono }
\newcommand{\AnggotaUjianA}{Dr. Tubagus Maulana Kusuma, SKom., MEngSc.}
\newcommand{\AnggotaUjianB}{Dr. rer. nat. I Made Wiryana, SSi., SKom.,MAppSc}
\newcommand{\AnggotaUjianC}{Dr. Lintang Yuniar Bonowosari, S.Kom., MSc.}
\newcommand{\Ringkasan}{Tulis ringkasan skripsi, pi, atau apa dengan bahasa yang jelas, lugas dan menggambarkan secara singkat tulisan ini.  Sebaiknya tidak lebih dari 150 kata dan sudah menjelaskan dari permasalahan, pembahasan dan penutup.}
\newcommand{\JumlahPustaka}{- -}
\newcommand{\JumlahHalaman}{125}
\newcommand{\JumlahHalamanDepan}{xiv}
\newcommand{\TahunPustaka}{1972-2013}
%%
%% Keterangan administratif sidang sarjana
%%
\newcommand{\TanggalSidang}{24 February 2015}
\newcommand{\TanggalLulus}{24 February 2015}
\newcommand{\TanggalSah}{}
\newcommand{\PejabatBagianSidang}{Dr. Edi Sukirman, S.Si, MM}
\setlength{\headheight}{15pt}

\pagestyle{fancy}
\renewcommand{\chaptermark}[1]{\markboth{Chapter  \thechapter.\ #1}{}}
\renewcommand{\sectionmark}[1]{\markright{\thesection.\ #1}{}}


\fancyhf{}
\fancyhead[LE,RO]{\thepage}
\fancyhead[RE]{\textit{\nouppercase{\leftmark}}}
\fancyhead[LO]{\textit{\nouppercase{\rightmark}}}

\fancypagestyle{plain}{ %
\fancyhf{} % remove everything
\fancyfoot[C]{\thepage}
\renewcommand{\headrulewidth}{0pt} % remove lines as well
\renewcommand{\footrulewidth}{0pt}}
\usepackage[margin=10pt,font=small,labelfont=bf,labelsep=endash]{caption}

\makeatother

\usepackage{listings}
\lstset{backgroundcolor={\color{bcolor}},
basicstyle={\ttfamily\small\color{codesty}},
breaklines=true,
commentstyle={\color{commandsty}},
deletekeywords={alter},
firstnumber=1,
frame=tb,
identifierstyle={\color{black}},
keywordstyle={\color{blue}\bfseries},
language=C,
morekeywords={dosync,if},
numberfirstline=true,
numbers=left,
numberstyle={\color{lineno}\sffamily\scriptsize},
showspaces=false,
showstringspaces=false,
stringstyle={\itshape\color{stringsty}}}
\addto\captionsamerican{\renewcommand{\lstlistingname}{Listing}}
\addto\captionsbahasa{\renewcommand{\lstlistingname}{Listing}}
\renewcommand{\lstlistingname}{Listing}

\begin{document}
\begin{onehalfspace}
\thispagestyle{empty}
\sloppy
\end{onehalfspace}

\pagenumbering{roman}

\vspace*{10mm}

\begin{center}
\textbf{\huge{}GUNADARMA UNIVERSITY}
\par\end{center}{\huge \par}

\begin{center}
\textbf{\MakeUppercase{\Fakultas}}
\par\end{center}

\vspace*{12mm}

\begin{center}
\includegraphics[width=35mm]{\string"gambar ika/univgundar2\string".jpg}
\par\end{center}

\vspace*{6mm}

\begin{center}
\textbf{\Large{}\JudulInggris }
\par\end{center}{\Large \par}

\vspace*{1cm}

\noindent \begin{center}
\begin{tabular}{lcl}
Name & : & \Name\tabularnewline
Student ID Number & : & \NPM\tabularnewline
Department & : & \Department\tabularnewline
Supervisor & : & \Supervisor\tabularnewline
\end{tabular}
\par\end{center}

 \vspace*{10mm}

\begin{center}
{\large{}Thesis Submitted to The Faculty of Industrial Technology}
\par\end{center}{\large \par}

\begin{center}
{\large{}Gunadarma University}
\par\end{center}{\large \par}

\begin{center}
{\large{}In Partial Fulfillment of The Requirements}
\par\end{center}{\large \par}

\begin{center}
{\large{}For Undergraduate Degree}
\par\end{center}{\large \par}

\begin{center}
{\large{}Jakarta}
\par\end{center}{\large \par}

\begin{center}
{\large{}\vspace*{3mm}
2015}
\par\end{center}{\large \par}


\chapter*{Statement of Originality and Publications}

\begin{singlespace}
\thispagestyle{empty}

Here by :
\end{singlespace}

\vspace*{0.5cm}

\begin{singlespace}
\begin{tabular}{>{\raggedright}p{35mm}c>{\raggedright}p{0.6\textwidth}}
Name & : & \textbf{\Name}\tabularnewline
Student ID Number & : & \textbf{\NPM}\tabularnewline
Title of Undergraduate Thesis & : & \textbf{\MakeUppercase{\JudulInggris}}\tabularnewline
Session Date & : & \textbf{\TanggalSidang}\tabularnewline
Passing Date & : & \textbf{\TanggalLulus}\tabularnewline
\end{tabular}
\end{singlespace}

\vspace*{0.5cm}

States that the writings are the result of my own work and may be
published entirely by Gunadarma University. All quotations in any
form have been following the rules and ethics that applicable. All
copyrights of logos and products mentioned in this book are the property
of their respective rights holders, unless otherwise noted. Regarding
the content and writing is the responsibility of authors, not Gunadarma
University. Hence this statement is made actually with full awareness.

\begin{flushright}
\vspace*{15mm}
\par\end{flushright}

\begin{flushright}
\Kota, \  \Tanggal\ \Bulan\ \Tahun
\par\end{flushright}

\begin{flushright}
\vspace*{15mm}
\par\end{flushright}

\begin{flushright}
(\Name)
\par\end{flushright}


\chapter*{Validation Page}

\thispagestyle{empty}

\begin{center}
Advisory Comitee\\
Session Date : \TanggalSidang
\par\end{center}

\begin{table}[h]
\begin{tabular}{|p{1cm}|p{8cm}|p{4cm}|}
\hline No. & Name & Position \\
\hline 1.  & \Supervisors & Supervisor \\
\hline 2.  & \AnggotaPembimbingA & Member \\
\hline 3.  & \AnggotaPembimbingB & Member \\
\hline
\end{tabular}
\end{table}

\begin{center}
Board of Examiners\\
Passing Date: \TanggalLulus
\par\end{center}

\begin{table}[h]
\begin{tabular}{|p{1cm}|p{8cm}|p{4cm}|}
\hline No. & Name & Position \\
\hline 1.  & \KetuaUjian & Chairman \\
\hline 2.  & \SekUjian & Secretary \\
\hline 3.  & \AnggotaUjianA & Member \\
\hline 4.  & \AnggotaUjianB & Member \\
\hline 5.  & \AnggotaUjianC & Member \\
\hline
\end{tabular}
\end{table}

\begin{center}
\begin{tabular}{c>{\centering}p{2cm}c}
\multicolumn{3}{>{\centering}p{135mm}}{\textbf{Acknowledged by ,}}\tabularnewline
\textbf{\footnotesize{}Supervisor } &  & \textbf{\footnotesize{}Department Head of Bachelor's }\tabularnewline
{\footnotesize{}~} &  & \textbf{\footnotesize{}Defense Examination}\tabularnewline
~ &  & ~\tabularnewline
~ &  & ~\tabularnewline
~ &  & ~\tabularnewline
~ &  & ~\tabularnewline
{\footnotesize{}(\Supervisor)} &  & {\footnotesize{}(\PejabatBagianSidang)}\tabularnewline
\end{tabular}
\par\end{center}


\chapter*{Abstract}

\begin{singlespace}
\addcontentsline{toc}{chapter}{Abstract}

\noindent\textbf{\Name. }\NPM

\noindent\textbf{\MakeUppercase{\JudulInggris}.} Undergraduate thesis,
\Fakultas, \Department Department, Gunadarma University, \Tahun.

\medskip{}

\end{singlespace}

\noindent Keyword : \emph{\KeyWords}

\begin{singlespace}
\medskip{}


\noindent (\JumlahHalamanDepan + \JumlahHalaman + appendix)

\bigskip{}

\end{singlespace}

The Java developer population are 9,007,346 making Java the top used
programming language in the world. With so many users of Java, computer
and internet allow the act of plagiarism against the Java. Since Internet
is accessible to everyone, it is easy to use Internet as a source
of information. However, copying document from Internet can be considered
as plagiarism. One of an algorithm to detect plagiarism is Levenshtein
Distance algorithm. Levenshtein Distance algorithm can calculate two
strings of arbitrary length and allow insertion, deletion also substitution.
This thesis is about how the implementation of the Levenshtein Distance
algorithm to determine the degree of similarity of two java source
codes and detect the presence of a source code plagiarism (as target
document) to another source code (as source document) in an application
called Detector of Java or Dj. This application can perform check
java source code in the local disk of user pc that will automatically
store into database. Besides, this application can also perform check
java source code stored in the database. To facilitate the use of
this application, there is short video tutorial that show step-by-step
usage and also tooltip that provide information about the use of the
features. In this application development process, first is a requirements
analysis of the functional and non-functional requirements. Second
is design with the help of UML and last is programming with the help
of NetBeans, MySQL and XAMPP. In testing process is conducted black-box
testing method of existing features.

\begin{singlespace}
\bigskip{}


\noindent References (\TahunPustaka)
\end{singlespace}


\chapter*{Acknowledgements}

\addcontentsline{toc}{chapter}{Acknowledgements}

First of all I would like to thank to Jesus Christ for His mercy and
blessings that always accompany me, so I be able to finish my undergraduate
thesis which entitled\JudulInggris. This undergraduate thesis is
intended to complete the requirement to finish my study in the Informatics
Engineering Department, Gunadarma University. The completion of this
thesis would not have been possible without the support and encouragement
from many people. Therefore, I would like to say thank to:
\begin{enumerate}
\item Prof. Dr. E. S. Margianti, SE., MM. Rector of Gunadarma University. 
\item Prof. Suryadi Harmanto, SSi, MMSI, 2nd Rector Assistant of Gunadarma
University who gave author the opportunity to join SARMAG Program. 
\item Prof. Dr. Ir. Bambang Suryawan, MT., Dean of Faculty of Industrial
Technology. 
\item Dr.-Ing. Adang Suhendra, SSi, SKom, MSc, head of Informatics Engineering
Program 
\item Drs. Haryanto, MMSI. director of SARMAG program who gave author a
chance to join SARMAG program.
\item Dr. Edi Sukirman, S.Si, MM as Department Head of Bachelors Defense
Examination.
\item 7. Dr. Lintang Yuniar Banowosari, SKom, MSc, as author\textquoteright s
supervisor. For her patience, guidance, support and encouragements
during the research. 
\item Remi Senjaya, ST, MMSI, as SarMag coordinator for support and help
author during SarMag period
\item All SarMag lecturers, for their knowledges and experiences given to
the author.
\item Author\textquoteright s beloved Father, Hilman Dollar Siregar, Author\textquoteright s
Mother, Corry Das Parlajoan Siregar, Author\textquoteright s Brother,
Deboner Hillery Hasiholan Siregar, and Author\textquoteright s Sister,
Icha Tifany Siregar, who always provide prayers, advices, support
and encourage the author to complete this thesis.
\item Author\textquoteright s friends in SMTI06 who always comfort, support,
and sharing their knowledge in this period. 
\item Author\textquoteright s friends in SarMag and Gunadarma University
who support the author and willing to answer questionnaire. 
\item Author\textquoteright s friends for support the author in completing
this thesis.
\end{enumerate}
This thesis is still need some improvement. Therefore, author is looking
forward for constructive criticism or suggestions. Hopefully, this
thesis can bring advantages for another authors and the readers.

\begin{flushright}
\Kota, \  \Tanggal\ \Bulan\ \Tahun
\par\end{flushright}

\begin{flushright}
\vspace*{15mm}
\par\end{flushright}

\begin{flushright}
(\Name)
\par\end{flushright}

\tableofcontents{}\listoffigures


\listoftables



\chapter{Introduction}

\selectlanguage{bahasa}%
\setcounter{page}{1}
\pagenumbering{arabic}

\selectlanguage{american}%

\section{\label{sec:Background}Background}

A computer is a partner of human life. It brings changes in every
sphere of human activity. Computer can process millions of instructions
in a few seconds, hence a computer can be defined as an automatic
electronic machine for performing calculations or controlling operations
that are expressible in numerical or logical terms. \citep{Elango}. 

In order to comprehend human aims, the programming language is created.
Programming Language is a language used to write instructions that
can be translated into machine language. One of the levels of computer
programming language is high-level languages. The advantages of using
a high-level language instead of others are simpler, more understandable,
consist of English language and does not depend upon machine. According
to IEEE Spectrum Java as one of high-level language is the most popular
programming languages followed by C, C++,C\# and Python. \citep{StephenCass}
The Java developer population are 9,007,346 making Java the top used
programming language in the world as reported by The 2009 Global Developer
Population and Demographics Survey, conducted by Evans Data Corporation.
\citep{Numjava}. 

Java is a high-level programming language originally developed by
Sun Microsystems and released in 1995. Java runs on a variety of platforms,
such as Windows, Mac OS, and the various version of UNIX. Java provides
advantages over other languages like object oriented, platform independent,
simple, secure, architectural-neutral, portable, robust, multithreaded,
interpreted, high performance, distributed and dynamic.\citep{TutorialJava}. 

With so many users of Java, computer and internet allows the act of
plagiarism against the Java source code. Since Internet is accessible
to everyone, it is easy to use Internet as a source of information.
However, copying documents from Internet can be considered as plagiarism.
It can even result to some legal problems, such as copyright infringement.\citep{Meichelbecks}
A new study of Pew Research Center shows that American college students
are increasingly plagiarizing others work and the Internet has played
a significant role in this trend.\citep{Anugrah}Plagiarism is an
academic crime and violation of ethics. Thus, plagiarism is essentially
a corruption of ethics. In education, plagiarism is a disgrace which
is not easily forgotten.\citep{Sugiyanto}. 

According to the data on the site Plagiarism.org, there is a study
that conducted by The Center for Academic Integrity with the result
that almost 80\% of students admitted to plagiarism at least one time.
Likewise, the results of a survey conducted by the Psychological Record,
36\% of students admitted to plagiarism against written documents.
\citealt{Regina}. In 1977 the people start show concerns about plagiarism
in source code. A survey in 2002 shows that 85.4\% in a class of 137
students at Monash University and 69.3\% in a class of 150 students
at Swinburne University are involved in source code plagiarism and
they admit this dishonesty. \citep{Ameera}. 

An easy way for plagiarism a source code is using Naked Plagiarism
or Semi Naked. Naked Plagiarism is a copy-paste techniques whereas
Semi Naked is add a few changes after doing a copy-paste . \citep{Michaelsen}.
Generally, plagiarism in coding is hard to be detected because of
the similar coding used for the same application. Plagiarism in coding
is easy to do, but difficult to detect (as cited in N. Wagner, 2000).
Students copy all or part of a program from some source or from different
sources and submit the copy as their own work. This includes students
who collaborate and submit similar work. Such plagiarism is felt to
be common, though the true extent is hard to assess. When a teacher
in a programming course gives same assignment problems to all students
then all students have to work on same problems. So it is the possibility
that some students write source code of problems by their own and
remaining students just take the code from them and amend it like
changing of variable names, changing the order of statements, functions
and variables of class and submit it. These types of modifications
in source code are very difficult to catch. \citep{Ahmad}

Some algorithms commonly used to detect plagiarism are Hamming Distance,
Longest Common Subsequence, Smith-Waterman, Needleman-Wunsch Algorithm
and Levenshtein Distance Algorithm. Hamming Distance allows only substitution,
hence, it only applies to strings of the same length. The Longest
Common Subsequence metric allows only insertion and deletion. Unlike
Hamming Distance and Longest Common Subsequence, Levenshtein Distance
can calculate two strings of arbitrary length and allow insertion,
deletion also substitution.

The levenshtein algorithm is used in some Translation Environment
Tools, such as translation memory leveraging applications to measure
the edit distance between two fuzzy matching content segments. The
Levenshtein distance is a metric for measuring the amount of difference
between two sequences. The Levenshtein distance between two strings
is given by the minimum number of operations needed to transform one
string into the other, where an operation is an insertion, deletion,
or substitution of a single character.\citep{Frederic} In this study
the author using Levenshtein Distance as algorithm that implemented
in application built using Java, to detect an act of plagiarism in
Java source program.


\section{\label{sec:Objectives}Objectives}

The aims of this study are:
\begin{enumerate}
\item Develop an application by implement the Levenshtein Distance algorithm
for detecting the similarity of the source code.
\item Knowing the process of detecting the similarity of the source code. 
\item Designing plagiarism detection application by using the Java Programming
Language in NetBeans. 
\item Checking a very large source code quickly and effectively. 
\end{enumerate}

\section{\label{sec:Scope-of-the-Work}Scope of the Work}

This thesis has limitation problem, that are : 
\begin{enumerate}
\item Extension of source code used in this study is java. 
\item Application considers all the code that is tested already free from
error.
\item The result is a percentage of how much similarity the target code
with source code 
\item The application works offline.
\end{enumerate}

\section{\label{sec:Research-Method}Research Method}

In this study required hardware and software. Minimum requirements
of hardware is a personal computer unit complete with specifications
are an Intel Pentium 4 (2 GHz), 512 MB RAM while requirement of software
are Lyx and Netbeans IDE 7.3. This study requires several steps: 
\begin{enumerate}
\item Theoretical Study. It is conducted by author by collecting knowledge
about plagiarism detection and Levenshtein Distance algorithm. 
\item Literature Study. It is conducted by author by collecting knowledge
about previous works by other researchers. 
\item Interview. It is conducted by author by collecting the results of
the interview several candidate of users.
\item System Design. It is conducted by author by implementing the theories,
some literatures and the results of the interview to design the application. 
\item Implementation. It is conducted by author by implementing the design
into code and constructing the application. 
\item Testing and Evaluation. It is conducted by author by test an application
that has been made to find errors then evaluated. 
\end{enumerate}

\section{\label{sec:Thesis-Structure}Thesis Structure}

Systematic of writing in this study is divided into five chapters,
namely Introduction, Theoretical Background, Analysis and Design,
Testing and Implementation, and Conclusion.
\begin{enumerate}
\item Introduction contains Background Issue, Objective of the Thesis, Scope
of the Work, Research Method, and Structure of the Thesis. 
\item Theoretical Background contains the relevant theories used in this
thesis. This theory involves understanding the application, the Java
language, Levenshtein algorithms and supporting software. 
\item Analysis and Design explain analysis requirements and design application
that have been performed. 
\item Testing and Implementation describes the implementation of the application
and application testing that has been completed. 
\item Conclusion describes conclusion of algorithms implementation that
has been done and add some suggestions regarding development for the
future. 
\end{enumerate}
\begin{onehalfspace}

\chapter{Theoretical Background}
\end{onehalfspace}


\section{\label{sec:Plagiarism}Plagiarism }

Plagiarism is derived from the Latin word 'plagiarius', which means
'kidnapper' from The Concise Oxford Dictionary. Plagiarism is defined
by S. Hannabuss as \textquotedblleft is the act of imitating or copying
or using somebody else\textquoteright s creation or idea without permission
and presenting it as one\textquoteright s own.\textquotedblright{}
\citep{Ahmad}. Hawley wrote that plagiarism is \char`\"{}perhaps
best conceptualised as existing along a continuum of behaviours ranging
from sloppy paraphrasing to the intentional copying of someone else's
work verbatim without credit to the source\char`\"{}. \citep{Christopher}.


\subsection{\label{sub:Type-of-Plagiarism}Type of Plagiarism}

There are five types of Plagiarism among student \citep{Barnbaum}
: 
\begin{enumerate}
\item Type I: Copy \& Paste Plagiarism Any time you lift a sentence or significant
phrase intact from a source, you must use quotations marks and reference
the source. 
\item Type II: Word Switch Plagiarism If you take a sentence from a source
and change around a few words, it is still plagiarism. If you want
to quote a sentence, then you need to put it in quotation marks and
cite the author and article. But quoting source articles should only
be done if what the quote says is particularly useful in the point
you are trying to make in what you are writing. In the case below,
a quotation would not be useful. The person who plagiarized in this
example has just been too lazy to synthesize the ideas expressed in
the source article. 
\item Type III: Style Plagiarism (This is trap that most students fall into.)
When you follow a source article sentence-by-sentence or paragraph-by-paragraph,
it is plagiarism, even though none of your sentences is exactly like
those in the source article or even in the same order. What you are
copying in this case, is the author's reasoning style. 
\item Type IV: Metaphor Plagiarism Metaphors are used either to make an
idea clearer or give the reader an analogy that touches the senses
or emotions better than a plain description of the object or process.
Metaphors, then, are an important part of an author's creative style.
If you cannot come up with your own metaphor to illustrate an important
idea, then use the metaphor in the source article, but give the author
credit for it.
\item Type V: Idea Plagiarism If the author of the source article expresses
a creative idea or suggests a solution to a problem, the idea or solution
must be clearly attributed to the author. Many students have difficulty
distinguishing an author's ideas and/or solutions from public domain
information. Public domain information is any idea or solution about
which people in the field accept as general knowledge. 
\end{enumerate}
According to Lars R. Jones, Phd. from Florida Institute of Technology,
Plagiarism among student is divided into four categories \citep{Lars}:
\begin{enumerate}
\item Category 1 - Unauthorized and/or unacknowledged collaborative work:
While students are expected to do their own research and writing,
instructors also understand that students may discuss their own research
projects with other students in the same course. Instructors strongly
suspect collaborative plagiarism when the same or similar phrases,
quotations, sentences, and/or parallel constructions appear in two
or more papers on the same topic. To protect yourself, you should
acknowledge\textemdash in a footnote or endnote\textemdash any significant
discussions you have had with others, as well as any advice, comments,
or suggestions that you have received from others, including your
instructor or other instructors if appropriate. 
\item Category 2 - Attempting to pass off, as your own work, a whole work
or any part of a work belonging to another person, group or institution:
This includes borrowing, buying, commissioning, copying, receiving,
downloading, taking, using, and/or stealing a paper that is not your
own. Submitting an entire work which is not your own also constitutes
research or academic fraud. A good rule of thumb is that original
student work should comprise at least 80 percent of any written assignment.
Assignments should not be padded out with Internet-harvested, borrowed,
paraphrased, quoted, or plagiarized material. Within this category,
four specific types of plagiarism can be identified : 

\begin{itemize}
\item Using such material without any attribution, citation, acknowledgment
and/or without quotation marks is plagiarism: You must use quotation
marks on any amount of text taken directly from another source, even
from the course textbook; moreover, such material must be cited correctly.
\item The use of such material with false attributions/citations and/or
the use of deceptive or fabricated citations to disguise direct plagiarism
is still plagiarism. Students who intentionally plagiarize often attempt
to disguise the plagiarized material in their papers with fake or
inadequate citations. This is especially true of Internet URLs which
can prove time-consuming to verify. Links to password-protected, private,
and fee-for-service (non-academic) domains\textemdash and even apparent
misspellings\textemdash should be considered suspect. Numerous false
and misleading citations within a paper may also constitute research
or academic fraud. Any citation must point the reader to the exact
location of the cited material, whether in print, in an on-line source,
or in another medium. 
\item The use of such material with quotation marks but without any attribution,
citation, or with inadequate/improper attribution/citation is considered
plagiarism. You must use proper citations for all quoted and paraphrased
material taken from another source. For example, the student used
quotation marks and seems to cite the quoted text but, by neglecting
to refer to the page from which this quotation was taken, has failed
to cite properly. 
\item The use of such material\textemdash correctly attributed and properly
cited\textemdash but without quotation marks is plagiarism. You must
use quotation marks on any amount of text taken from another source.
For example, the student cited material that was copied, in large
part, directly from the source text but the student failed to indicate
the quoted material by using quotation marks. The student pretended
to be paraphrasing but was really plagiarizing. 
\end{itemize}
\item Category 3 - The use of any amount of text that has been improperly
paraphrased constitutes plagiarism. Suggesting an improper reliance
on a single source, this includes \textquotedblleft mosaic plagiarism\textquotedblright{}
or \textquotedblleft cut-and-paste plagiarism.\textquotedblright{}
To paraphrase improperly is simply to put the words of a source text
in a different order or form while retaining the main idea that is
the intellectual property of the original author/translator. When
you simply alter the text itself (but not the author\textquoteright s
idea), all that you have done is to eliminate the obvious need for
quotation marks; you have not eliminated the need for an explanatory
citation/attribution. The idea itself remains the intellectual property
of the original author/translator and, therefore, must be cited as
such. 
\item Category 4 - The use of any amount of text, that is properly paraphrased\textemdash but
which is either not cited or which is improperly cited\textemdash constitutes
plagiarism. This includes papers in which a general failure to cite
sources or a gross negligence in citing sources is apparent. Moreover,
attaching false, misleading, or improper attributions/citations to
properly paraphrased texts still constitutes plagiarism. 
\end{enumerate}

\subsection{\label{sub:Plagiarisme-Cases}Plagiarisme Cases}

Plagiarism of written text and software code can occur in many areas
and is not simply limited to academic institutions. For example, in
1988 Barbara Chase-Riboud brought a plagiarism lawsuit against the
film director Steven Spielberg and Dreamworks SKG over their latest
film called Amistad. Chase-Riboud argued that the characters, scenes
and other aspects of her book entitled Echo of Lions had been illegally
copied without her permission. However, on February 9th 1988 Chase-Riboud
dropped the charges and the film was released in December. \citep{Doug}.
Another examples of famous plagiarism cases are \citep{Joe} :
\begin{enumerate}
\item 1. Harry Potter and Willy the Wizard In June 2009, the estate of deceased
author Adrian Jacobs sued the publishers of Harry Potter. They alleged
that JK Rowling\textquoteright s fourth book in the Potter series,
The Goblet of Fire, lifted heavily from Jacobs\textquoteright{} previously
published novel The Adventures of Willy the Wizard: Livid Land. Both
main characters go through a series of challenges in order to save
friends. The estate first sued Bloomsbury, the UK publisher, and when
after the American publisher, Scholastic, in 2011. Judges in both
cases dismissed the claims for lack of evidence, but Jacob\textquoteright s
son and grandson who run the estate may try to take the case to Australia. 
\item The Wild Blue and Wings of Morning Historian and writer Stephen Ambrose
came under plagiarism fire in 2002 when his latest work, The Wild
Blue: The Men and Boys Who Flew the B-234s Over Germany was compared
to a 1995 book Wings of Morning: The Story of the Last American Bomber
Shot Down Over Germany in World War II. Although there was likely
to be some cross over because of the similar subject matter, exact
phrases seem to be copied from the older version of the book. The
case was never taken to trial, but journalists found additional cases
of possible plagiarism in other books by Ambrose. 
\item Jayson Blair and the New York Times As a staff reporter for the New
York Times, Jayson Blair committed journalistic fraud on numerous
occasions. The NYT broke their own story in 2003 after investigation
into Blair\textquoteright s 600 articles showed plagiarism from other
newspaper stories or downright lies. Blair reported on news without
actually being there, invented quotes from sources and created events
out of thin air. After the investigation, Blair resigned and the New
York Times published an apology for their oversight and asked readers
to help continue the investigation. 
\item Joe Biden and Neal Kinnock Before Joe Biden became Vice President,
he was a senator from Delaware who wanted to be president. During
the 1987 presidential race, Biden\textquoteright s stump speech in
Iowa concluded with several phrases lifted from a speech delivered
by Neal Kinnock, a politician from the UK. Challenger Michael Dukakis
picked up on the incident and created an attack video that showed
the similarities between the speeches. Although Biden later admitted
that he had been inspired by the speech and had given it reference
in previous uses of it, this time was different. New York Times reporter
Maureen Dowd broke the story and brought it national attention in
the pre-YouTube world. Biden ended up pulling out the 1988 race entirely. 
\item Maureen Dowd and Talking Points Speaking of Maureen Dowd, Dowd herself
was accused of plagiarism in 2009. Over 20 years after she broke Biden\textquoteright s
story, an online Sunday column on the New York Times website showed
striking similarities to a story on the Talking Points Memo website.
There was a word-for-word paragraph in both with just small change.
Her defense? A friend who helped her write the column must have seen
the Talking Points Memo post and \textquotedblleft accidentally\textquotedblright{}
integrated it into Dowd\textquoteright s column. 
\item George Harrison and The Chiffons George Harrison\textquoteright s
first post-Beatles single \textquotedblleft My Sweet Lord\textquotedblright{}
was released in 1971. Although the song hit \#1 on the charts, it
bore striking music resemblance to \textquotedblleft He\textquoteright s
So Fine\textquotedblright{} recorded by the Chiffons and released
in 1962. It didn\textquoteright t take long for the Chiffons to sue
for copyright infringement. The suit included George Harrison, Apple
Records, BMI and Hansen Publications and finally went to court in
1976. The judge found that Harrison did not intend to plagiarize but
may have done so subconsciously. Harrison had to pay the Chiffons
\$587,000. 
\item Coldplay and Joe Satriani Coldplay\textquoteright s 2008 song \textquotedblleft Viva
la Vida\textquotedblright{} was a hit, but also had a very similar
chord progression to a song by guitarist Joe Satriani. Satriani\textquoteright s
song \textquotedblleft If I Could Fly\textquotedblright{} was released
in 2004. The day after Coldplay received nominations for seven Grammy
awards, Satriani filed suit. The court dismissed the case without
prejudice in 2009. 
\end{enumerate}

\subsection{\label{sub:Factor-to-Plagiarise}Factor to Plagiarise}

Bennett conducted a detailed literature review on factors motivating
students to plagiarise and classified these factors into five categories
\citep{Georgina}:
\begin{enumerate}
\item Means and opportunity: the fact that resources are readily available
and easily accessible by students over the Internet makes it convenient
for students to gain instant and easy access to large amounts of information
from many sources. Furthermore, many Internet sites exist that provide
essays to students, and also many of these sites provide chargeable
services to students for writing their essays and papers. 
\item Personal traits: factors such as the desire to gain high marks, fear
of failure, pressure from family members, desire to achieve future
job prospects and career development, and competitive behaviour amongst
students can cause them to plagiarise. 
\item Individual circumstances: students who work to finance their studies
have less time to study and time pressures can cause them to cheat. 
\item Academic integration: students that do not integrate or \textquoteleft fit
in\textquoteright{} with University life, and are therefore alienated
from their courses or instructors, are more likely to cheat than students
who are better integrated in University life. 
\item Other factors: these include the student\textquoteright s previous
experiences of plagiarism prior to attending higher education, their
education experience of developing good writing skills, their nationality
(i.e. foreign students with language difficulties are more likely
to cheat than home students fluent in the language). Other factors
include age and gender. 
\end{enumerate}

\subsection{\label{sub:Plagiarism-in-Source}Plagiarism in Source Code}

Survey result by Georgina Cosma and Mike Joy of source-code plagiarism
in programming assignments is plagiarism can occur when a student
refuses source-code authored by someone else and, intentionally or
unintentionally, fails to acknowledge it adequately, thus submitting
it as his/her own work. This involves obtaining the source-code, either
with or without the permission of the original author, and reusing
source-code produced as part of another assessment (in which academic
credit was gained) without adequate acknowledgement. \citep{GeorginaMike}.
\begin{enumerate}
\item Reusing 

\begin{enumerate}
\item Reproducing or copying source code without making any alterations. 
\item Reproducing or copying source-code and adapting it minimally or moderately;
minimal or moderate adaptation occurs when the source-code authored
by someone else. 
\item Converting all or part of someone else\textquoteright s source-code
to a different programming language may constitute plagiarism, depending
on the similarity between the languages and the effort required by
the student to do the conversion. Conversion may not constitute plagiarism
if the student borrows ideas and inspiration from source-code written
in another programming language and the source-code is entirely authored
by the student. 
\item Generating source-code automatically by using code-generating software;
this could be construed as plagiarism if the use of such software
is not explicitly permitted in the assignment specification. 
\end{enumerate}
\item Obtaining 


Obtaining the source-code either with or without the permission of
the original author includes the following : 
\begin{enumerate}
\item Paying another individual to create a part of or all their source-code 
\item Stealing another student\textquoteright s source-code 
\item Collaborating with one or more students to create a programming assignment
which required students to work individually, resulting in the students
submitting similar source-codes; such inappropriate collaboration
may constitute plagiarism or collusion (the name of this academic
offense varies according to the local academic regulations). 
\item Exchanging parts of source-code between students in different groups
carrying out the same assignment with or without the consent of their
fellow group members.
\end{enumerate}
\item Inadequately Acknowledging 


Inadequately acknowledging source-code authorship includes the following
: 
\begin{enumerate}
\item Failing to cite the source and authorship of the source-code, within
the program source-code (in the form of an-in-text citation within
a comment) and in the appropriate documentation. 
\item Providing fake references (i.e., references that were made-up by the
student and that do not exist); this is a form of academic offense,
often referred to as fabrication, which may co-occur wih plagiarism. 
\item Providing falsereferences (i.e., references exist but do not match
the source-code that was copied); another form of academic offense,
often reffered to as falsification, which may co-occur with plagiarism. 
\item Modifying the program output to make it seem as if the program works
when it is not working; this too is a form of academic offense akin
to falsification, which may co-occur with plagiarism. 
\end{enumerate}
\end{enumerate}
According to Jones \citet{Georgina}, \textquotedblleft A plagiarized
program is either an exact copy of the original, or a variant obtained
by applying various textual transformations\textquotedblright . Jones
provides the following list of transformations:
\begin{enumerate}
\item Verbatim copying, 
\item Changing comments, 
\item Changing white space and formatting, 
\item Renaming identifiers, 
\item Reordering code blocks, 
\item Reordering statements within code blocks, 
\item Changing the order of operands/operators in expressions, 
\item Changing data types, 
\item Adding redundant statements or variables, and 
\item Replacing control structures with equivalent structures. 
\end{enumerate}

\section{\label{sec:Java}Java}

James Gosling initiated the Java language project in June 1991 for
use in one of his many set-top box projects. The language, initially
called Oak after an oak tree that stood outside Gosling's office,
also went by the name Green and ended up later being renamed as Java,
from a list of random words. Sun released the first public implementation
as Java 1.0 in 1995. It promised Write Once, Run Anywhere (WORA),
providing no-cost run-times on popular platforms. On 13 November 2006,
Sun released much of Java as free and open source software under the
terms of the GNU General Public License (GPL). On 8 May 2007, Sun
finished the process, making all of Java's core code free and open-source,
aside from a small portion of code to which Sun did not hold the copyright.
\citep{TutorialJava} .


\subsection{\label{sub:Characters-of-Java}Characters of Java}

Java has many characters such as : 
\begin{enumerate}
\item Object Oriented: In Java, everything is an Object. Java can be easily
extended since it is based on the Object model. 
\item Platform independent: Unlike many other programming languages including
C and C++, when Java is compiled, it is not compiled into platform
specific machine, rather into platform independent byte code. This
byte code is distributed over the web and interpreted by virtual Machine
(JVM) on whichever platform it is being run. 
\item Simple: Java is designed to be easy to learn. If you understand the
basic concept of OOP, Java would be easy to master. 
\item Secure: With Java's secure feature, it enables to develop virus-free,
tamper-free systems. Authentication techniques are based on public-key
encryption. 
\item Architectural-neutral: Java compiler generates an architecture-neutral
object file format, which makes the compiled code to be executable
on many processors, with the presence of Java runtime system. 
\item Portable: Being architectural-neutral and having no implementation
dependent aspects of the specification makes Java portable. Compiler
in Java is written in ANSI C with a clean portability boundary which
is a POSIX subset. 
\item Robust: Java makes an effort to eliminate error prone situations by
emphasizing mainly on compile time error checking and runtime checking. 
\item Multithreaded: With Java's multithreaded feature, it is possible to
write programs that can do many tasks simultaneously. This design
featu\citet{TutorialJava}re allows developers to construct smoothly
running interactive applications. 
\item Interpreted: Java byte code is translated on the fly to native machine
instructions and is not stored anywhere. The development process is
more rapid and analytical since the linking is an incremental and
lightweight process. 
\item High Performance: With the use of Just-In-Time compilers, Java enables
high performance. 
\item Distributed: Java is designed for the distributed environment of the
internet. 
\item Dynamic: Java is considered to be more dynamic than C or C++ since
it is designed to adapt to an evolving environment. Java programs
can carry extensive 
\end{enumerate}

\section{\label{sec:Tools}Tools }

The tools used in the making this application as follows : 


\subsection{\label{sub:Unified-Modeling-Language}Unified Modeling Language }

The Unified Modeling Language (UML) is a graphical notation for drawing
diagrams of software concepts. One can use it for drawing diagrams
of a problem domain, a proposed software design, or an already completed
software implementation. \citep{Robert}.


\subsubsection{\label{sub:Use-Case-Diagram}Use Case Diagram }

A use case is a description of the behavior of a system. That description
is written from the point of view of a user who has just told the
system to do something particular. A use case captures the visible
sequence of events that a system goes through in response to a single
user stimulus. A visible event is an event that the user can see.
\citep{James}. Table \ref{tab:Notation-of-Use} shows the notation
of use case diagram which consists of seven notation they are actor,
use case, interaction, dependency, association, generalization, and
realization. \citep{TutorialUML}.

\begin{table}[!tph]
\begin{centering}
\protect\caption{\label{tab:Notation-of-Use}Notation of Use Case Diagram }

\par\end{centering}

\centering{}%
\begin{tabular}{|>{\centering}p{1cm}|>{\centering}m{4cm}|>{\centering}m{7cm}|}
\hline 
No & Notation & Description\tabularnewline
\hline 
\hline 
1. & \includegraphics[scale=0.4]{\string"gambar ika/actor\string".png} & An actor can be defined as some internal or external entity that interacts
with the system.\tabularnewline
\hline 
2. & \includegraphics[scale=0.5]{\string"gambar ika/usecase\string".png} & Use Case represents a set of actions performed by a system for a specific
goal.\tabularnewline
\hline 
3. & \includegraphics[scale=0.7]{\string"gambar ika/interaction\string".png} & Interaction is defined as a behavior that consistsof a group of messages
exchanged among elements to accomplish a specific task.\tabularnewline
\hline 
4. & \includegraphics[scale=0.7]{\string"gambar ika/dependecy\string".png} & Dependency is a relationship between two things in which change in
one element also affects the other one.\tabularnewline
\hline 
5. & \includegraphics[scale=0.7]{\string"gambar ika/association\string".png} & Association is basically a set of links that connects elements of
an UML model. It also describes how many objects are taking part in
that relationship.\tabularnewline
\hline 
6. & \includegraphics[scale=0.7]{\string"gambar ika/generalization\string".png} & Generalization can be defined as a relationship which connects a specialized
element with a generalized element. It basically describes inheritance
relationship in the world of objects. \tabularnewline
\hline 
7. & \includegraphics[scale=0.7]{\string"gambar ika/realization\string".png} & Realization can be defined as a relationship in which two elements
are connected. One element describes some responsibility which is
not implemented and the other one implements them. \tabularnewline
\hline 
\end{tabular}
\end{table}



\subsubsection{\label{sub:Class-Diagram}Class Diagram}

A class diagram is a graphic presentation of the static view that
shows a collection of declarative (static) model elements, such as
classes, types, and their contents and relationships. A class diagram
may show a view of a package and may contain symbols for nested packages.
A class diagram contains certain reified behavioral elements, such
as operations.\citep{James}. Table \ref{tab:Notation-of-Class} consists
of three columns : number, notation and description about notation
of class diagram. \citep{TutorialUML}.

\begin{table}[!tph]


\protect\caption{\label{tab:Notation-of-Class}Notation of Class Diagram }


\begin{centering}
\begin{tabular}{|>{\centering}p{1cm}|>{\centering}m{5cm}|>{\centering}m{6cm}|}
\hline 
No & Notation & Description\tabularnewline
\hline 
\hline 
1. & \includegraphics[scale=0.4]{\string"gambar ika/class\string".png} & Class represents set of objects having similar responsibilities.\tabularnewline
\hline 
2. & \includegraphics[scale=0.7]{\string"gambar ika/interaction\string".png} & Interaction is defined as a behavior that consistsof a group of messages
exchanged among elements to accomplish a specific task.\tabularnewline
\hline 
3. & \includegraphics[scale=0.7]{\string"gambar ika/generalization\string".png} & Generalization can be defined as a relationship which connects a specialized
element with a generalized element. It basically describes inheritance
relationship in the world of objects.\tabularnewline
\hline 
\end{tabular}
\par\end{centering}

\end{table}



\subsection{\label{sub:Flowchart}Flowchart}

Flowcharts are usually drawn using some standard symbols; however,
some special symbols can also be developed when required. Some standard
symbols, which are frequently required for flowcharting many computer
programs are shown in Table \ref{tab:Flowchart} \citep{Mumbai}.

\begin{table}[!tph]
\protect\caption{\label{tab:Flowchart}Flowchart }


\begin{centering}
\begin{tabular}{|>{\centering}p{1cm}|>{\centering}m{5cm}|>{\centering}m{6cm}|}
\hline 
No & Flowchart & Description\tabularnewline
\hline 
\hline 
1. & \includegraphics[scale=0.7]{\string"gambar ika/1\string".png} & Terminator: An oval flow chart shape indicates the start or end of
the process, usually containing the word \textquotedblleft Start\textquotedblright{}
or \textquotedblleft End\textquotedblright .\tabularnewline
\hline 
2. & \includegraphics[scale=0.7]{\string"gambar ika/2\string".png} & Process: A rectangular flow chart shape indicates a normal/generic
process flow step. For example : \textquotedblleft Add 1 to X\textquotedblright ,
\textquotedblleft M = M{*}F\textquotedblright{} or similar.\tabularnewline
\hline 
3. & \includegraphics[scale=0.7]{\string"gambar ika/3\string".png} & Decision: A diamond flow chart shape indicates a branch in the process
flow. This symbol is used when a decision needs to be made, commonly
a Yes/No question or True/False test.\tabularnewline
\hline 
4. & \includegraphics[scale=0.7]{\string"gambar ika/4\string".png} & Connector: A small, labelled, circular flow chart shape used to indicate
a jump in the process flow. Connectors are generally used in complex
or multi-sheet diagrams.\tabularnewline
\hline 
5. & \includegraphics[scale=0.7]{\string"gambar ika/5\string".png} & Data: A parallelogram that indicates data input or output (I/O) for
a process. Examples: Get X from the user, Display X.\tabularnewline
\hline 
6. & \includegraphics[scale=0.7]{\string"gambar ika/6\string".png} & Delay: used to indicate a delay or wait in the process for input from
some other process.\tabularnewline
\hline 
7. & \includegraphics[scale=0.7]{\string"gambar ika/7\string".png} & Arrow: used to show the flow of control in a process. An arrow coming
from one symbol and ending at another symbol represents that control
passes to the symbol the arrow points to.\tabularnewline
\hline 
\end{tabular}
\par\end{centering}

\end{table}



\subsection{\label{sub:Levenshtein-Algorithm}Levenshtein Algorithm}

The Levenshtein Distance Algorithm (LD) measure the similarity between
source string(s) and target string (t).The similarity between two
strings is measured as distance between source and target string.
Number of substitutions, insertion and deletion required to convert
source strong into target string are referred as difference between
these files. The Levenshtein Algorithm result increase with difference
between the strings. If source string(s) is \char`\"{}fellow\char`\"{}
and target string (t) is \char`\"{}follow\char`\"{} then LD=1. It
means that one substitution is required to change source string into
target string. If s is \char`\"{}this thesis is related to plagiarism\char`\"{}
and t is \char`\"{}the thesis is about plagiarism\char`\"{}, then
LD (s,t) = 10. Levenshtein distance was used in Plagiarism Detection.
It is also been used in Spell Checking, Speech recognition and DNA
Analysis. \citep{Ahmad}.


\subsubsection{\label{sub:The-Levenshtein-Algorithm}The Levenshtein Algorithm }

An algorithm of Levenshtein Distance divided into seven steps : 
\begin{enumerate}
\item \textquotedblleft Set n to be the length of s. 


Set m to be the length of t.


If n = 0, return m and exit. 


If m = 0, return n and exit. 

\item Initialize the first row to 0..n. 


Initialize the first column to 0..m. 

\item Examine each character of s (i from 1 to n). 
\item Examine each character of t (j from 1 to m). 
\item If s{[}i{]} equals t{[}j{]}, the cost is 0. 


If s{[}i{]} does not equal t{[}j{]}, the cost is 1. 

\item Set cell d{[}i,j{]} of the matrix equal to the minimum of: 

\begin{enumerate}
\item The cell immediately above plus 1: d{[}i-1,j{]} + 1. 
\item The cell immediately to the left plus 1: d{[}i,j-1{]} + 1. 
\item The cell diagonally above and to the left plus the cost: d{[}i-1,j-1{]}
+ cost. 
\end{enumerate}
\item After the iteration steps (3, 4, 5, 6) are complete, the distance
is found in cell d{[}n,m{]}.\textquotedblright{} 
\end{enumerate}

\subsection{\label{sub:Netbeans}Netbeans}

Two kinds of netbeans are Netbeans IDE and Netbeans Platform. The
NetBeans IDE is an open-source integrated development environment,
it supports development of all Java application types (Java SE including
JavaFX, (Java ME, web, EJB and mobile applications) out of the box.
Among other features are an Ant-based project system, Maven support,
refactorings, version control (supporting CVS, Subversion, Mercurial
and Clearcase). The NetBeans Platform is a reusable framework for
simplifying the development of Java Swing desktop applications. The
NetBeans IDE bundle for Java SE contains what is needed to start developing
NetBeans plugins and NetBeans Platform based applications; no additional
SDK is required. \citep{Rahul}.


\subsubsection{\label{sub:Advantages-of-Netbeans}Advantages of Netbeans IDE}

The NetBeans integrated development environment (IDE) delivers. The
NetBeans IDE can boost a productivity when working with Java SE, Java
EE, or Java ME technology as well as PHP, Groovy, JavaScript, and
C/C++. The advantages of using Netbeans IDE also the reasons to use
the NetBeans IDE are \citep{Netbeans} :
\begin{enumerate}
\item Works Out of the Box Simply download and install the NetBeans IDE
and you are good to go. Installation is a breeze with its small download
size. All IDE tools and features are fully integrated\textemdash no
need to hunt for plug-ins\textemdash and they work together when you
launch the IDE. 
\item Free and Open Source When you use the NetBeans IDE, you join a vibrant,
open-source community of thousands of users ready to help and contribute.
There are discussions on the NetBeans mailing lists, blogs on PlanetNetBeans,
and helpful FAQs. 
\item Connected Developer The NetBeans IDE is the tool of choice for teams
working in a collaborative environment. You can create and manage
java.net-hosted projects, for example; file issue tracking reports
using both Jira and Bugzilla, and collaborate with like-minded developers\textemdash all
directly from within the familiar interface of the IDE. 
\item Powerful GUI Builder The GUI Builder (formerly known as Project Matisse)
supports a sophisticated yet simplified Swing Application Framework
and Beans Binding. Now you can build GUIs in a natural way. 
\item Support for Java Standards and Platforms The IDE provides end-to-end
solutions for all Java development platforms including the latest
Java standards. 

\begin{enumerate}
\item Java Mobility Support Complete environment to create, test, and run
applications for mobile devices. With preprocessor blocks, you can
readily handle fragmentation issues. Support for Java Mobility development
is the best among all Java development tools.
\item Java Enterprise Edition (EE) 6 support: The first free, open-source
IDE to support Java EE 6 specifications. 
\item Java Standard Edition (SE) Support: You can develop applications using
the latest Java SE standards. 
\end{enumerate}
\item Profiling and Debugging Tools With NetBeans IDE profiler, you get
realtime insight into memory usage and potential performance bottlenecks.
Furthermore, you can instrument specific parts of code to avoid performance
degradation during profiling. The HeapWalker tool helps you evaluate
Java heap contents and find memory leaks. 
\item Dynamic Language Support The NetBeans IDE provides integrated support
for scripting languages such as PHP, Groovy, and JavaScript. 

\begin{enumerate}
\item PHP: With the NetBeans IDE for PHP, you get the best of both worlds:
the productivity of an IDE (code completion, real-time error checking,
debugging and more) with the speed and simplicity of your favorite
text editor in a less than 30mb download. 
\item JavaScript: The NetBeans IDE has the JavaScript tools you need: an
intelligent JavaScript editor, CSS/HTML code completion, the ability
to debug JavaScript in Firefox and IE, and bundled popular JavaScript
libraries. Your favorite JavaScript framework will get you 80\% of
the way, NetBeans IDE will help you with that last 20\%. 
\item Groovy: In the NetBeans IDE, you can now create Grails applications,
integrate Groovy scripts with your JavaSE project. 
\end{enumerate}
\item Extensible Platform Start with its extensible platform and add your
own NetBeans IDE features and extensions or build an IDE-like application,
keeping only features you want. Extending the platform and its Swing-based
foundation saves development time and can optimize performance. 
\item Customizable Projects Through the NetBeans IDE build process, which
relies on industry standards such as Apache Ant, make, Maven, and
rake, rather than a proprietary build process, you can easily customize
projects and add functionality. You can build, run, and deploy projects
to servers outside of the IDE. 
\item Non-Java Code Support You're not limited to the Java programming language.
You can include many other programming languages, such as C/C++, scripting
languages like JavaScript, etc. Even more exciting, define your own
language and include it in your projects. 
\item Dedicated Support Available When you can't get the help you need from
the community, consider Developer Support Packages, which offer programming
advice, software support, and training credits.
\end{enumerate}

\subsubsection{\label{sub:Netbeans-IDE-versi}Netbeans IDE versi 7.3 }

NetBeansTM IDE is a modular, standards-based integrated development
environment (IDE), written in the JavaTM programming language. The
NetBeans project consists of a full-featured open source IDE written
in the Java programming language and a rich client application platform,
which can be used as a generic framework to build any kind of application.
Figure \ref{fig:Netbeans-IDE-7.3} show the appearance of Netbeans
IDE 7.3. \citep{Netbeans}

\begin{figure}[!tph]
\includegraphics[scale=0.39]{\string"gambar ika/netbeans\string".png}

\centering{}\protect\caption{\label{fig:Netbeans-IDE-7.3}Netbeans IDE 7.3}
\end{figure}
NetBeans IDE 7.3 supports the technologies in Table \ref{tab:Supported-Technologies-of}
and has been tested with application servers described in Table 2.5
: 

\begin{table}[!tph]
\protect\caption{\label{tab:Supported-Technologies-of}Supported Technologies of Netbeans}


\centering{}%
\begin{tabular}{|>{\centering}p{2cm}|>{\centering}p{8cm}|}
\hline 
\multicolumn{2}{|>{\centering}p{10cm}|}{Supported Technologies}\tabularnewline
\hline 
\hline 
1. & Java EE 7, Java EE 6, Java EE 5 and J2EE 1.4\tabularnewline
\hline 
2. & JavaFX 2.2.x\tabularnewline
\hline 
3. & Java ME SDK 3.2\tabularnewline
\hline 
4. & Java Card 3 SDK\tabularnewline
\hline 
5. & Struts 1.3.10\tabularnewline
\hline 
6. & Spring 3.1, 2.5\tabularnewline
\hline 
7. & Hibernate 3.2.5\tabularnewline
\hline 
8. & Java API for RESTful Web Services (JAX-RS) 1.1\tabularnewline
\hline 
9. & Java Wireless Toolkit 2.5.2 for CLDC\tabularnewline
\hline 
10. & Issue Tracking\tabularnewline
\hline 
11. & Bugzilla 4.0.x and earlier\tabularnewline
\hline 
12. & Git 1.7.\textcyr{\char245}\tabularnewline
\hline 
13. & Jira 5.0 and earlier\tabularnewline
\hline 
14. & PHP 5.4, 5.3, 5.2, 5.1\tabularnewline
\hline 
15. & Groovy 1.8.6\tabularnewline
\hline 
16. & Grails 2.0\tabularnewline
\hline 
17. & Apache Ant 1.8.4\tabularnewline
\hline 
18. & Apache Maven 3.0.4 and earlier\tabularnewline
\hline 
19. & C/C++/Fortran\tabularnewline
\hline 
20. & VCS\tabularnewline
\hline 
21. & Subversion: 1.7.x, 1.6.x\tabularnewline
\hline 
22. & Mercurial: 2.2.x and earlier\tabularnewline
\hline 
23. & ClearCase V7.0\tabularnewline
\hline 
\end{tabular}
\end{table}


\begin{table}[!tph]


\protect\caption{Tested Application Servers of Netbeans}


\begin{centering}
\begin{tabular}{|>{\centering}p{2cm}|>{\centering}p{8cm}|}
\hline 
\multicolumn{2}{|>{\centering}p{10cm}|}{Tested Application Servers}\tabularnewline
\hline 
\hline 
1. & GlashFish Server Open Source Edition 4.0\tabularnewline
\hline 
2. & WebLogic 12c\tabularnewline
\hline 
3. & Jboss AS 7\tabularnewline
\hline 
\end{tabular}
\par\end{centering}

\end{table}



\subsection{\label{sub:MySQL}MySQL}

MySQL is a fast, easy-to-use Relational DataBase Management System
(RDMS) being used for many small and big businesses. MySQL is developed,
marketed, and supported by MySQL AB, which is a Swedish company. \citep{TutorialMYSQL}.
MySQL is becoming so popular because of many good reasons: 
\begin{enumerate}
\item MySQL is released under an open-source license. 
\item MySQL is a very powerful program in its own right. It handles a large
subset of the functionality of the most expensive and powerful database
packages. 
\item MySQL uses a standard form of the well-known SQL data language. 
\item MySQL works on many operating systems and with many languages including
PHP, PERL, C, C++, JAVA, etc. 
\item MySQL works very quickly and works well even with large data sets. 
\item MySQL is very friendly to PHP, the most appreciated language for web
development. 
\item MySQL supports large databases, up to 50 million rows or more in a
table. The default file size limit for a table is 4GB, but you can
increase this (if your operating system can handle it) to a theoretical
limit of 8 million terabytes (TB). 
\item MySQL is customizable. The open-source GPL license allows programmers
to modify the MySQL software to fit their own specific environments. 
\end{enumerate}

\subsection{\label{sub:JDBC}JDBC}

JDBC stands for Java Database Connectivity, which is a standard Java
API for database-independent connectivity between the Java programming
language and a wide range of databases. \citep{TutorialJDBC}. The
JDBC library includes APIs for each of the tasks commonly associated
with database usage: 
\begin{enumerate}
\item Making a connection to a database 
\item Creating SQL or MySQL statements 
\item Executing that SQL or MySQL queries in the database 
\item Viewing \& Modifying the resulting records 
\end{enumerate}

\subsection{PhpMyAdmin}

PhpMyAdmin is a free software tool written in PHP, intended to handle
the administration of MySQL over the Web. phpMyAdmin supports a wide
range of operations on MySQL, MariaDB and Drizzle. Frequently used
operations (managing databases, tables, columns, relations, indexes,
users, permissions, etc) can be performed via the user interface,
while user still have the ability to directly execute any SQL statement.
\citep{PhpMyAdmin}.


\subsection{\label{sub:XAMPP}XAMPP}

XAMPP is a small and light Apache distribution containing the most
common web development technologies in a single package. Its contents,
small size, and portability make it the ideal tool for students developing
and testing applications in PHP and MySQL. XAMPP is available as a
free download in two specific packages: full and lite. \citep{Dalibor}. 


\section{\label{sec:Existing-Plagiarism-Detector}Existing Plagiarism Detector}

An example of existing plagiarism detector of source code are \citep{Ahmad}
: 
\begin{enumerate}
\item SIM (Software Similarity Tester) SIM is used to detect plagiarism
of code written in Java, C, Pascal, Modula-2, Lisp, and Miranda. SIM
is also used to check similarity between plain text files. SIM converts
the six source code into strings of token and then compare these strings
by using dynamic programming string alignment technique. This technique
is also used in DNA string matching. The alignment is very expensive
and exhaustive computationally for all applications because for large
code repositories SIM is not scalable. The source code of SIM is available
publically but it is no more actively supported. 
\item MOSS (Measure of Software Quality) MOSS is available free to use in
academics and it is accessible as an online service. Moss support
Ada programs, Java, C, C++, plain text and Pascal. At the same time
MOSS also support UNIX and windows operating systems. First of all
MOSS convert source code into tokens and then use robust winnowing
algorithm. Robust Winnowing Algorithm is introduced by Schleimeretal
but the internal detail of working of this algorithm is confidential.
This algorithm takes the document fingerprints by selecting a subset
of token hashes. In the comparing process of set of files, \textquotedblleft MOSS
creates an inverted index to map document fingerprints to documents
and their positions within each document. Next, each program file
is used as a query against this index, returning a list of documents
in the collection having fingerprints in common with the query.\textquotedblright{}
The number of matching fingerprints of each pair of document in the
set of files is the result of MOSS. MOSS sort these results and show
highest-score matches to user. 
\item JPlag JPlag is available publically as free accessible service. JPlag
can be used to check plagiarism of source code written in C, C++,
Scheme and Java. First of all source code in the directory is parsed
and then transformed into token strings. After transformation JPlag
compare these strings by using Running Karp-Rabin Greedy String Tiling
(RKR-RGST) algorithm. The comparison result then shown in HTML file,
which can be visited by using any browser. In the HTML file of results
main page consist of pairs of programs that are assume to be plagiarized.
User can see results of different pairs separately. Different fonts
in result file of HTML shows different things, like the pairs with
similar code will have different font from other pairs. In this way
user can analyze results very easily. JPlag has been used extensively
by various academic institutions, both at the undergraduate and the
graduate level, reaching the submissions as many as 500 participants
since fall 2001, receiving and processing dozens of submission each
month. 
\end{enumerate}
\begin{onehalfspace}

\chapter{Analysis and Design}
\end{onehalfspace}


\section{\label{sec:General-Description}General Description}

Current plagiarisim detection of source code manually is by reading
and comparing the source code line by line as shown in Figure \ref{fig:Check-File-Manually}.

\begin{figure}[!tph]
\begin{centering}
\includegraphics{\string"gambar ika/manual\string".png}
\par\end{centering}

\protect\caption{\label{fig:Check-File-Manually}Check File Manually}
\end{figure}


The biggest obstacle faced is how if the source code to be tested
consists of hundreds or even thousands of lines? Surely, it will take
a long time, besides one of human nature is a decreased level of concentration
as time goes by. So that errors can occur when testing a large source
code. The purpose of this application is to help users to test the
source code quickly and accurately.


\section{\label{sec:Analysis-of-Requirements}Analysis of Requirements}

At the stage of the analysis requirements will analyze all requirements
and what user expected from the application. Requirements are divided
into two types, they are functional requirements and non-functional
requirements.


\subsection{\label{sub:Functional-Requirements}Functional Requirements}

Data collection techniques for functional requirements of users in
this plagiarism detection named Dj for Detector of Java is obtained
by distributing questionnaires to candidate users, so the features
that are expected to be obtained . One of the results obtained is
largely users are already familiar with the Java programming language.
Other results are shown in Table \ref{tab:Table-3.1-Functional} that
is a list of features that are obtained from the distribution of questionnaires.
\begin{table}[!tph]
\begin{centering}
\protect\caption{\label{tab:Table-3.1-Functional}Table 3.1 Functional Requirements}

\par\end{centering}

\noindent \centering{}%
\begin{tabular*}{12cm}{@{\extracolsep{\fill}}|c|>{\centering}m{4cm}|>{\raggedright}m{5cm}|}
\hline 
\multirow{1}{*}{No } & Feature & \centering{}Information\tabularnewline
\hline 
\hline 
1. & Input a Threshold & \centering{}User can input a value of threshold.\tabularnewline
\hline 
2. & Save in Database  & \centering{}Files .java that tested will be saved automatically in
the database. \tabularnewline
\hline 
3. & Menu to Search & \centering{}Searching file by name or size. \tabularnewline
\hline 
4. & Menu to Delete & \centering{}Delete file from database. \tabularnewline
\hline 
5. & Showing the Content of File & \centering{}Showing content of two files, source code and target code. \tabularnewline
\hline 
\end{tabular*}
\end{table}


All the features contained in Table 3.1 will be found on this Dj application. 


\subsection{\label{sub:Non-Functional-Requirements}Non Functional Requirements}

Beside of functional requirements, an analysis of the non-functional
requirements that are needed by the application also performed. One
of non-functional requirements of these application is user friendly,
furthermore is described in Table \ref{tab:Table-3.2-Non-Functional}
a list of the minimum hardware and software.

\begin{table}[!tph]


\protect\caption{\label{tab:Table-3.2-Non-Functional}Table 3.2 Non-Functional Requirements}


\centering{}%
\begin{tabular}{|c|c|c|}
\hline 
No & Hardware & Software\tabularnewline
\hline 
\hline 
1. & Processor Intel Pentium 4 & Microsoft Windows 7 \tabularnewline
\hline 
2. & Memory 2 GB & NetBeans IDE 7 \tabularnewline
\hline 
3. & - & XAMPP \tabularnewline
\hline 
4. & - & JDBC\tabularnewline
\hline 
\end{tabular}
\end{table}



\section{\label{sec:Application-Development}Application Development}

Development in this Dj application is to implement the Levenshtein
Distance algorithm. But previously done some pre-processing stage.
The stages performed in this application can be seen in Figure \ref{fig:Steps-in-Development}.

\begin{figure}[!tph]
\begin{centering}
\includegraphics[scale=0.5]{Dia/stepDevelopment}
\par\end{centering}

\protect\caption{\label{fig:Steps-in-Development}Steps in Development Stage}
\end{figure}


The first third stages contained in Figure \ref{fig:Steps-in-Development}
is a pre-processing stage. Next will be processed by the Levenshtein
Distance algorithm and final stage is comparing the result with the
threshold value.


\subsection{\label{sub:Delete-Comments}Delete Comments}

In this stage, the application will look for comments on both of source
code and target code by finding the \char`\"{}\textbackslash{}\textbackslash{}\char`\"{}
or \char`\"{}\textbackslash{} {*} .. {*} \textbackslash{}\char`\"{}
then delete it as described in Figure \ref{fig:Algorithm-of-Delete}.

\begin{figure}[!tph]
\begin{centering}
\includegraphics[scale=0.5]{Dia/deleteComment}
\par\end{centering}

\protect\caption{\label{fig:Algorithm-of-Delete}Algorithm of Delete Comment in Application}


\end{figure}



\subsection{\label{sub:Rename-All-Words}Rename All Words except Reserved Words}

There is a list of reserved words in Java, this application will search
for all the words besides the reserved words listed then replace it
with another word, but in this application will be replaced with \textquotedblleft aa\textquotedblright{}
then rewrite in new file. Figure \ref{fig:Algorithm-of-Rename} described
the algorithm of this stage.

\begin{figure}[!tph]
\begin{centering}
\includegraphics[scale=0.5]{Dia/Rename}
\par\end{centering}

\protect\caption{\label{fig:Algorithm-of-Rename}Algorithm of Rename All Words Except
Reserved Words}


\end{figure}



\subsection{\label{sub:Sort-by-Size}Sort by Size}

The next stage is sorting by size, this stage will sort functions
or methods according to the size from smallest to largest. Figure
\ref{fig:Algorithm-of-Sort} described that the end of this stage,
all sorted method will be rewritten.

\begin{figure}[!tph]
\begin{centering}
\includegraphics[scale=0.5]{Dia/sortBySize}
\par\end{centering}

\protect\caption{\label{fig:Algorithm-of-Sort}Algorithm of Sort Methods}


\end{figure}



\subsection{\label{sub:Implementation-Levenshtein-Dista}Implement Levenshtein
Distance}

This stage is the stage of implement of the Levenshtein Distance algorithm
to the source code and the target code, but before getting to this
stage, both of file has been modified into a string. The Levenshtein
Distance Algorithm (LD) measure the similarity between source string
(s) and target string (t). The similarity between two strings is measured
as distance between source string and target string. Figure \ref{fig:Algorithm-of-Levenshtein}
shows a Levenshtein Distance algorithm \citet{Ahmad}.

\begin{figure}[!tph]
\begin{centering}
\includegraphics[scale=0.35]{Dia/levenshteinDistance}
\par\end{centering}

\protect\caption{\label{fig:Algorithm-of-Levenshtein}Algorithm of Levenshtein Distance}
\end{figure}


The goal is to calculate the similarity of two strings. As we know
that Levenshtein Distance algorithm gives a difference between two
strings. In Figure \ref{fig:Algorithm-of-Levenshtein} the result
is a distance called Diff, which will be used in the following formula
to get plagiarized value with Max(CS,ST) is the larger length value
of either of target string or of source string. 

\begin{center}
Plagiarized Value =$\left\{ 1-\frac{Diff}{Max(CS,ST}\right\} *100$
\par\end{center}


\subsection{\label{sub:Compare-with-the}Compare with the Threshold}

Figure \ref{fig:Algorithm-of-Compare} is the last stage, it compares
the result obtained from the four previous stages with the threshold
value entered by the user. If the threshold is greater than the result
then the target code is not considered as plagiarism, in contrast
the target code is plagiarism.

\begin{figure}[!tph]
\begin{centering}
\includegraphics[scale=0.5]{Dia/compareWithThreshold}
\par\end{centering}

\protect\caption{\label{fig:Algorithm-of-Compare}Algorithm of Compare Result with
Threshold}
\end{figure}



\section{\label{sec:Application-Design}Application Design}

This application design using UML (Unified Modeling Language) and
Flowchart Diagram.


\subsection{\label{sub:Unified-Modeling-Language-1}Unified Modeling Language}

Unified Modeling Language (UML) used in this application design are
Use Case Diagram and Class Diagram.


\subsubsection{\label{sub:Use-Case-Diagram-1}Use Case Diagram Design}

A use case is a description of the behavior of a application. That
description is written from the point of view of a user who has just
told the application to do something particular. In Figure \ref{fig:Use-Case-Diagram}
explained that user can manipulate file and view file. User can add
a file, test a file and delete file from database. User also can search
a file in database.

\begin{figure}[!tph]
\begin{centering}
\includegraphics[scale=0.7]{Dia/usecase}
\par\end{centering}

\protect\caption{\label{fig:Use-Case-Diagram}Use Case Diagram of Dj}
\end{figure}


This application is intended for users who want to test keplagiatan
a particular document lecturer programming and programmers.


\subsubsection{\label{sub:Class-Diagram-Design}Class Diagram Design}

A class diagram is a graphic presentation of the static view that
shows a collection of declarative (static) model elements, such as
classes, types, and their contents and relationships. A class diagram
may show a view of a package and may contain symbols for nested packages.
Class Diagram of Dj application described in Figure \ref{fig:Class-Diagram-of}.

\begin{figure}[!tph]
\includegraphics[scale=0.3]{\string"Dia/class Diagram\string".png}

\protect\caption{\label{fig:Class-Diagram-of}Class Diagram of Dj}
\end{figure}


Figure\ref{fig:Class-Diagram-of} shows that there are 13 classes
in this application. PlagiarismDetectionLD as a main class will call
MainMenu class which is an interface class. As shown in Figure\ref{fig:Class-Diagram-of}
arrows are directed from CheckPC class, CheckDB class and Processing
class to MainMenu class, it means MainMenu class has a relation with
them and also will call them to process file. When application try
to add or delete new file in database, MainMenu class will interacting
with Processing class which has interaction with ConnectDB class.
When application try to check plagiarism or not a file from PC then
MainMenu class will call CheckPC class, on the other hand MainMenu
class will interacting with CheckDB class when application try to
check file from Database (DB). While the steps described in Figure
\ref{fig:Steps-in-Development} they are DeleteComment, Rename, Sorting
and LevenshteinDistance are related to the Process class, which has
interaction with CheckPC and CheckDB class. The last class is Video.java
that has interaction with MainMenu class to show a short tutorial
video.


\subsection{\label{sub:Flowchart-Design}Flowchart Design}

Figure \ref{fig:Flowchart-of-Dj} shows that the user will determine
whether checking files from PC, checking files from Database, search
file, or delete file. In checking the files either from a PC or a
Database, the user must enter the threshold value. In conducting the
search, the user enters the name or size of the files to be searched.
When the user wants to delete the file, the user selects the file
in the table. All those activities will be responded by the application
by providing the results and display it in the table.

\begin{figure}[!tph]
\begin{centering}
\includegraphics[scale=0.5]{Dia/flowchartUmum}
\par\end{centering}

\protect\caption{\label{fig:Flowchart-of-Dj}Flowchart of Dj}
\end{figure}



\subsection{\label{sub:Table-Structure}Table Structure}

Table \ref{tab:Table-of-File} is a table that contains data in database
used in Dj application, which is named pd\_table. It contains name
field, size field and content field.

\begin{table}[!tph]
\protect\caption{\label{tab:Table-of-File}Table of File in Database}


\centering{}%
\begin{tabular}{|c|>{\centering}m{2cm}|c|>{\centering}m{6cm}|}
\hline 
No & Field Name  & Type & Comment\tabularnewline
\hline 
\hline 
1. & name  & Varchar (30)  & Name of file either of source code or target code. It must be primary.\tabularnewline
\hline 
2. & size & Varchar (8)  & Size of source code and target code in bytes.\tabularnewline
\hline 
3. & content & Varchar (10000) & Content of source code and target code.\tabularnewline
\hline 
\end{tabular}
\end{table}



\section{\label{sec:Interface-Design}Interface Design}

Interface design is a design that is made before the application is
made. Interface design aims to simplify the process of making and
development an application.


\subsection{\label{sub:Main-Menu-Design}Main Menu Design}

In the design of MainMenu class in this application there are a table,
About button, Search button, Delete button and Refresh button. Besides,
there is the menu bar named Check, Check menu bar contains two menu
items they are fromPC and fromDB that will connect MainMenu class
to the CheckPC class and CheckDB class as has been illustrated in
Figure \ref{fig:Class-Diagram-of} Diagram Class.

\begin{figure}[!tph]
\begin{centering}
\includegraphics[scale=0.7]{\string"gambar ika/mockup1\string".png}
\par\end{centering}

\protect\caption{\label{fig:Design-of-Main}Design of Main Menu}


\end{figure}


The table in Figure \ref{fig:Design-of-Main} will display the file
contained in the database that is the name and size, but the content
of the file does not appear to maintain the neatness of the application,
the button About contains information about the Dj application, the
Search button used to do a searching of file on the table, the Delete
button to delete file from table and database and the last is refresh
button to refresh the application.


\subsection{\label{sub:Checking-from-Personal}Checking from Personal Computer
Design}

Figure \ref{fig:Design-of-Checking} illustrates the display when
the user want to check the value of plagiarism files from the PC.
There are two Browse button to conduct a search of the Source Code
and Target Code. User input the threshold value in the text field
provided. Once the Process button is pressed, the content of Source
Code and Target Code will be displayed on two Text Area, and the third
Text Area will show the calculation result that will indicate whether
the Target Code plagiarism or not.

\begin{figure}[!tph]
\begin{centering}
\includegraphics[scale=0.7]{\string"gambar ika/mockup2\string".png}
\par\end{centering}

\protect\caption{\label{fig:Design-of-Checking}Design of Checking from PC}
\end{figure}



\subsection{\label{sub:Checking-activity-from}Checking activity from Database
Design}

Checking from Database\textquoteright s design in Figure \ref{fig:Design-of-Checking-1}
is not much different than the Checking from PC\textquoteright s design,
the difference is the selection of Source Code and Target Code is
not using Browse button but with combo box that will display the names
of the files that have been stored in the database.

\begin{figure}[!tph]
\begin{centering}
\includegraphics[scale=0.7]{\string"gambar ika/mockup3\string".png}
\par\end{centering}

\protect\caption{\label{fig:Design-of-Checking-1}Design of Checking from DB}
\end{figure}



\chapter{Testing and Implementation}


\section{\label{sec:Implementation}Implementation}

Implementation stage consists of the stages that have been described
previously in Chapter 3 into the program, database implementation
and interface design implementation.


\subsection{\label{sub:Database-Implementation}Database Implementation}

In chapter Analysis and Design has discussed database design. The
software used is XAMPP. In Figure \ref{fig:-Run-an} is shown the
XAMPP control panel. In this control panel makes sure that Apache
and MySQL are in the state of running.

\begin{figure}[!tph]
\begin{centering}
\includegraphics[scale=0.5]{\string"gambar ika/implementasi/xampp\string".PNG}
\par\end{centering}

\protect\caption{\label{fig:-Run-an} Run an Apache and MySQL in XAMPP Control Panel}
\end{figure}


In Figure \ref{fig:Create-Database-using} is the phpMyAdmin page,
where phpMyAdmin is used for create a database. Making the database
can be done by selecting the Databases menu then input the name of
the database that is plagiarism\_detection. Next create a table named
pd\_table with code \textquotedblleft Create table (name varchar(30)
NOT NULL, size varchar(8) NOT NULL, content varchar (10000) NOT NULL,
PRIMARY KEY (name));\textquotedblright{}

\begin{figure}[!tph]
\includegraphics[scale=0.4]{\string"gambar ika/implementasi/phpMyAdmin\string".PNG}

\centering{}\protect\caption{\label{fig:Create-Database-using}Create Database using phpMyAdmin}
\end{figure}



\subsection{\label{sub:Program-Implementation}Program Implementation}

Program Implementation is done by coding in NetBeans. As already described
in the previous chapter, this stage is divided into five steps they
are Delete Comment, Rename All Words except Keywords, Sort Method
by Size, Implementation of the Levenshtein Distance and Compare Result
with Threshold.


\subsubsection{\label{sub:Delete-Comment}Delete Comment}

deleteComment method in Listing \ref{lis:Program-Implementation-of}will
process contentFile1 and contentFile2 variable which type is String.
ContentFile1 and contentFile2 are the content of the source code and
the target code inputted or selected by the user. In List 4.1 fifth
and sixth line are to eliminate the comments by searching for signs
// and remove all that stands behind it or delete everything that
is between the signs / {*} with {*} /.

\begin{lstlisting}[caption={Program Implementation of Delete
Comment},label={lis:Program-Implementation-of}]
public String[] deleteComment (String contentFile1, String contentFile2) throws IOException
{                      
	content1 = contentFile1;              
	content2 = contentFile2;              
	content1=content1.replaceAll("(?:/\\*(?:[^*]|(?:\\*+[^*/]))*\\*+/)|(?://.*)","");//use regex            
	content2=content2.replaceAll("(?:/\\*(?:[^*]|(?:\\*+[^*/]))*\\*+/)|(?://.*)","");                         
	String files[] ={content1,content2};                       
	return files; //return content of new file 1 and file 2     
}
\end{lstlisting}


This deleteComment method returns a string array named files, which
contains content1 and content2 which are the source code and the target
code has no comment.


\subsubsection{\label{sub:Rename-All-Words-1}Rename All Words except Keywords}

In Listing \ref{lis:Rename-Words-except} there is a rename method
that processes the file string array variable. This method has an
array variable that contains a set of keywords in Java (Barry). file1
and file2 string variable containing the source code and the target
code that has been processed by the previous method in Listing \ref{lis:Program-Implementation-of}
named deleteComment, then the file will be separated by a space using
the code \char`\"{} .split (\char`\"{}\textbackslash{}\textbackslash{}
s +\char`\"{}); \char`\"{}. The program will read the file and if
it finds a word that is one of the keywords then written back directly
to the newfile string variable but if the word is not the keywords
it will be changed to \char`\"{}aa\char`\"{} and added to the variable
newfile. 

\begin{lstlisting}[caption={Rename Words except Keywords},label={lis:Rename-Words-except}]
public String [] rename(String [] file){     
	String keywords[]= {"abstract","assert","boolean","break","byte","case","catch","char","class","const","continue","default","do","double","else","enum","extends","final","finally","float","for","goto","if","implements","import","instanceof","int","interface","long","native","new","package","private","protected","public","return","short","static","strictfp","super","switch","synchronized","this","throw","throws","transient","try","void","volatile","while","const","goto","false","null","true","[","]","{","}","(",")","=",";","\\."," ","+","-",","};           
	
	file1=file[0];     
	file2=file[1];       
	String files1[]= file1.split("\\s+");  
	String files2[]= file2.split("\\s+");
	
	for (int i=0;i<files2.length;i++){             
		for (int j=0; j<keywords.length;j++ ){                 
			if (files2[i].contains(keywords[j])){                      
				newFile2=newFile2+" "+files2[i];                     
				break;                 
			}                 
			if (j+1==keywords.length) {                                     
				if(files2[i].matches("[(a-zA-Z.? )]*")){                         
					files2[i]="aa";                         
					newFile2=newFile2+" "+files2[i];                         
					break;                      
				}                 
		}             
	}         
}
\end{lstlisting}



\subsubsection{\label{sub:Sort-by-Size-1}Sort by Size}

SortFile method on the Listing \ref{lis:Collect-Curly-Brackets} process
a newFile string variable in Listing \ref{lis:Rename-Words-except}.
Listing \ref{lis:Collect-Curly-Brackets} will read a source file
and target file character by character and when it finds \char`\"{}\{\char`\"{},
the position of the sign will be stored on pos list as well as banyak
variable will be added 1. Banyak variable aims to calculate how many
\char`\"{}\{\char`\"{} in file. After reading the entire contents
of the file and count all the \char`\"{}\{\char`\"{}, the program
will look for pairs of each \char`\"{}\{\char`\"{} and write it on
the posbaru list. postbaru will be processed further by Sorting2 class.

\begin{lstlisting}[caption={Collect Curly Brackets},label={lis:Collect-Curly-Brackets}]
public String  sortFile(String file){       
	newFile=file;       
	int i;       
	String isiposbaru;       
	List<Integer> pos = new ArrayList<Integer>();        
	int banyak=0;        
	for (i=0;i<newFile.length();i++){             
		if (newFile.charAt(i)=='{'){                
			banyak++; //to count how many { sign                 
			pos.add(i);             
		}          
	}               
	List<String> posbaru = new ArrayList<String>();       
	StringBuffer sb = new StringBuffer();//for saving char between { and }         
	for(i=1;i<pos.size()+1;i++){ // start from { in the rear                 
		for ( int j=pos.get(pos.size() - i);j<newFile.length();j++){                    
			sb.append(newFile.charAt(j));                     
			if (newFile.charAt(j)=='}'){                         
				isiposbaru=sb.toString();                         
				posbaru.add(isiposbaru);                         
				newFile=newFile.replace(isiposbaru, ""); //to delete all method from string file                        
				 sb.delete(0, sb.length());//delete all element in string buffer sb                      
				 break;                     
			}            
		}        
	}                  
	Sorting2 sorting2 = new Sorting2();         
	String list = sorting2.sortingMethods(posbaru);        
	return list;                
}
\end{lstlisting}


compare method on the Listing \ref{lis:Sort-by-Size} is in a class
Sorting2, this method will compare string1 which is a collection of
code in a curly brackets with string2 which is also a collection of
code in a curly brackets. This method returns zero if they are equal.
It returns a positive value if string1 is greater than string2. Otherwise,
a negative value is returned. 

\begin{lstlisting}[caption={Sort by Size},label={lis:Sort-by-Size}]
public int compare (String string1, String string2) {                         
	return string2.length() - string1.length();        
}
\end{lstlisting}



\subsubsection{\label{sub:Levenshtein-Distance-Implementat}Levenshtein Distance
Implementation}

Listing \ref{lis:Implementation-Levenshtein-Dista} is the implementation
of the Levenshtein Distance algorithm to the Java language (rosettacode).
Levenshtein Distance described in Chapter 3 will be converted into
the Java language. It makes matrix with row is file1 and file2 as
column.

\begin{lstlisting}[caption={Implementation Levenshtein
Distance},label={lis:Implementation-Levenshtein-Dista}]
 public double levenshteinDistance(String [] file) {        
	file1 = file[0];        
	file2 = file[1];        
	file1 = file1.toLowerCase();         
	file2 = file2.toLowerCase();           
	int [] costs = new int [file2.length() + 1];       
	for (int j = 0; j < costs.length; j++)          
		 costs[j] = j;       
		 for (int i = 1; i <= file1.length(); i++) {           
			costs[0] = i;            
			int nw = i - 1;            
			for (int j = 1; j <= file2.length(); j++) {   
			//fill the matriks                
			int cj = Math.min(1 + Math.min(costs[j], costs[j - 1]), file1.charAt(i - 1) == file2.charAt(j - 1) ? nw : nw + 1); 
			// if same,then cj is min  (1 + Math.min(costs[j], costs[j - 1]),nw); else cj is min (1 + Math.min(costs[j], costs[j - 1]),nw+1);                 
			nw = costs[j];                 
			costs[j] = cj; //rewrite new array costs            
		}       
	 }         
}
\end{lstlisting}


List \ref{lis:Implementation-Levenshtein-Dista} matches word in the
column with the word in the line. If they match then put the value
\textquotedblleft Above Left\textquotedblright{} cell in the matching
cell. If they do not match then take minimum of three values and add
1 to it. The three values will be from cell above, above left and
left of current cell. \citet{Ahmad}.


\subsubsection{\label{sub:Compare-with-a}Compare with a Threshold}

Listing \ref{lis:Formula-of-Plagiarized} is the implementation of
a program to process the result of the previous stage of implementation
Levenshtein Distance. value variable in Listing \ref{lis:Formula-of-Plagiarized}
will contain the result of the following formula:

\begin{center}
Plagiarized Value =$\left\{ 1-\frac{Diff}{Max(CS,ST}\right\} *100$
\par\end{center}

\begin{lstlisting}[caption={Formula of Plagiarized Value in
Java},label={lis:Formula-of-Plagiarized}]
double value = (1-((double)(costs[file2.length()])/Math.max(file1.length(), file2.length())))*100;         
return value;
\end{lstlisting}


Result of the Listing \ref{lis:Formula-of-Plagiarized} is value variable,
later will called as result. Then in Listing \ref{lis:Comparing-Result-with}
will do a comparison between the result with a threshold which is
a user input. If the result is greater than or equal to the threshold,
the target code declared as plagiarism to the source code, if not
then the target code does not commit plagiarism.

\begin{lstlisting}[caption={Comparing Result with Threshold},label={lis:Comparing-Result-with}]
if (result>=threshold){                     
	jTextArea3.setText("Nilai Threshold yang dimasukkan: "+ threshold + " %" + "\nHasil yang didapat: "                         
	+ (String.format("%.02f", result)) + " %" + "\n>> " + file1+ " melakukan "                         
	+ "tindakan plagiarisme terhadap " + file2);                
	}                 
else {                     
	jTextArea3.setText("Nilai Threshold yang dimasukkan: "+ threshold + " %" + "\nHasil yang didapat: "                        
	+ (String.format("%.02f", result)) + " %" + "\n>> "+file1 + " tidak melakukan "                         
	+ "tindakan plagiarisme terhadap " + file2);                
}
\end{lstlisting}



\subsection{\label{sub:Interface-Design-Implementation}Interface Design Implementation}

In the implementation of the interface will explain how the appearance
of this application in accordance with the design that has been discussed
in Chapter 3. The user interface consists of a main menu interface,
check from pc interface and check from database interface, hereinafter
referred to db. 

In Figure \ref{fig:Old-Dj-Logo} is shown that the Dj (Detector of
Java) application logo which consists of the Java language logo itself
in the form of cup added with a signal representing detect action.
However, after the redesign can be seen in Figure \ref{fig:Dj-Logo}
Dj logo become the old logo with a disc which is used by DJ (Disc
Jockey) as a characteristic of Dj word in general. 

\begin{figure}[!tph]
\begin{centering}
\includegraphics{\string"gambar ika/implementasi/dj1\string".png}
\par\end{centering}

\protect\caption{\label{fig:Old-Dj-Logo}Old Dj Logo }
\end{figure}


\begin{figure}[!tph]
\begin{centering}
\includegraphics{\string"gambar ika/implementasi/dj\string".png}
\par\end{centering}

\protect\caption{\label{fig:Dj-Logo}Dj Logo}
\end{figure}



\subsubsection{\label{sub:Main-Menu-Interface}Main Menu Interface Design Implementation}

In the main menu interface design implementation as shown in Figure
\ref{fig:Main-Menu-Interface} is provided a table that contains the
name and the size of the files that are stored in the database. In
addition there is a Go to Check menu that consists of two items as
in Figure \ref{fig:Main-Menu-Interface-1} they are From PC for personal
computer and From DB for database, for each is useful as a bridge
to get into the check from PC interface and check from DB interface.
There are also a help, about, search, delete and refresh button in
accordance with the interface design.

\begin{figure}[!tph]
\begin{centering}
\includegraphics[scale=0.7]{\string"gambar ika/implementasi/mainmenu1\string".png}
\par\end{centering}

\protect\caption{\label{fig:Main-Menu-Interface}Main Menu Interface Design Implementation}
\end{figure}


\begin{figure}[!tph]
\begin{centering}
\includegraphics[scale=0.7]{\string"gambar ika/implementasi/mainmenu2\string".png}
\par\end{centering}

\protect\caption{\label{fig:Main-Menu-Interface-1}Main Menu Interface Design Implementation}
\end{figure}


In Figure \ref{fig:Main-Menu-Interface-1}displayed a tooltip contained
on Check From PC menu item. This Dj application equipped with a tooltip
that aims to facilitate users to use this application.


\subsubsection{\label{sub:Check-from-PC}Check from PC Interface Design Implementation}

Interface Design of check from PC in Figure \ref{fig:Check-from-PC}
in accordance with the previous design, which consists of two Browse
button for user select the file to be tested. The name of each file
will be displayed in the text field provided, while the third text
field will be inputted by the user desired threshold value. When the
Process button is clicked, contents both of file and the result will
be displayed in the text area provided.

\begin{figure}[!tph]
\begin{centering}
\includegraphics[scale=0.5]{\string"gambar ika/implementasi/dfrompc1\string".PNG}
\par\end{centering}

\protect\caption{\label{fig:Check-from-PC}Check from PC Interface Design Implementation}
\end{figure}



\subsubsection{\label{sub:Check-from-DB}Check from DB Interface Design Implementation}

Check from DB Interface Design in Figure \ref{fig:Check-from-DB}
in accordance with the previous design, which consists of two combo
box that contain the name of files stored in the database, however
the rest of the display is the same as check from PC interface design.

\begin{figure}[!tph]
\begin{centering}
\includegraphics[scale=0.5]{\string"gambar ika/implementasi/dfromdb1\string".PNG}
\par\end{centering}

\protect\caption{\label{fig:Check-from-DB}Check from DB Interface Design Implementation}
\end{figure}



\section{\label{sec:Testing}Testing}

Tests conducted on this application is the Black-Box testing method
is to observe the result of the execution through the test data that
focuses on functional requirements. The purpose of this test is to
determine whether the application is running according to the function
and find if there is an error in the application. 

The data used in this test is word.java and word2.java \citet{Ahmad}are
shown in Listing \ref{lis:Content-of-word.java} and Listing \ref{lis:Content-of-word1.java}. 

\begin{lstlisting}[caption={Content of word.java as Source Code},label={lis:Content-of-word.java}]
public class Word implements Comparable<Word>{ 	
private String word; 
	/** 	
	* To change this template, choose Tools | Templates 	
	* and open the template in the editor. 	
	*/ 	
	public Word(String str) { 		
		this.word = str; 	
	} 
	public String toString() { 		
		return word; 	
	} 
		
	public boolean equals (Object other) { 		
		if (this.compareTo((Word) other ==0) { 			
			return true; 		
		} else { 			
			return false; 		
		} 	
	}

	/** 	 
	* Compute and return a hashcode for the word. 	 
	* @return int hashcode 	
	*/ 	

	public int hashCode() {  		
		int hc=0; 		
		for ( int i=0;i<word.length();i++){ 			
			char c = word.charAt(i); 			
			hc += Character.getNumericValue(c); //ASCII number 		
		} 		
		return hc; 	
	} 
	/** 	
	* Compares two words lexicographically 	 
	* @param  	 
	*/ 	 
	public int compareTo(Word w) { 		
		return (word.compareToIgnoreCase(w.word)); 	
	} 
} 	
\end{lstlisting}


\begin{lstlisting}[caption={Content of word1.java as Source
Code},label={lis:Content-of-word1.java}]
public class Word implements Comparable<Word>{ 
	private String word; 
	public Word(String str) { 		
		word = str.toLowerCase(); 	
	} 

	public int hashCode() {  		
		int hc=0; 		
		for ( int i=0;i<word.length();i++){ 		
			char c = word.charAt(i); 			
			hc += Character.getNumericValue(c); //ASCII number 		
		} 		
		return hc; 
	} 

	public boolean equals (Object other) { 	
		if (this.compareTo((Word) other ==0) { 		
			return true; 	
		} else { 		
			return false; 		
		} 	
	} 

	public int compareTo(Word w) { 		
		return (word.compareToIgnoreCase(w.word)); 	
	} 
	
	public String toString() { 	
		return word; 	
	} 		 
}		 
\end{lstlisting}


In this scenario, word.java is the source code while word2.java is
the target code. It can be seen clearly that word2.java in Listing
\ref{lis:Content-of-word1.java} is plagiarism against word.java in
Listing \ref{lis:Content-of-word.java}. Target code removes some
of the comments and change some position methods. Tests performed
on the source code and the target code in this Dj are checking files
from pc then after files are stored in the database it will recheck
same files from db. Other testing is a search and deletion of files
as well as other additional features.


\subsection{\label{sub:Check-From-PC}Check From PC Testing}

This stage of testing was to determine whether the application can
test the user-selected files from the PC located in local disk D.
In Figure \ref{fig:Testing-of-Check} the application has display
the content of the source code, the target code and the result of
the calculation.

\begin{figure}[!tph]
\begin{centering}
\includegraphics[scale=0.5]{\string"gambar ika/uji/ujiPC1\string".png}
\par\end{centering}

\protect\caption{\label{fig:Testing-of-Check}Testing of Check from PC feature}
\end{figure}


Based on Figure \ref{fig:Testing-of-Check} it can be concluded that
the application can check plagiarism of files in PC. In Figure \ref{fig:word.java-and-word2.java}
shows that files are already stored on the database so another conclusion
is the application can save files that have been tested to the database
used by application.

\begin{figure}[!tph]
\begin{centering}
\includegraphics[scale=0.7]{\string"gambar ika/uji/ujiPC2\string".png}
\par\end{centering}

\protect\caption{\label{fig:word.java-and-word2.java}word.java and word2.java have
saved in Database}
\end{figure}



\subsection{\label{sub:Check-From-DB}Check From DB Testing}

This stage aims to determine whether the application is able to check
the files that already exist in the database. In Figure \ref{fig:Testing-of-Check-1}
application shows the content of word.java and word2.java from database
and also the result of calculation.

\begin{figure}[!tph]
\begin{centering}
\includegraphics[scale=0.5]{\string"gambar ika/uji/ujiDB\string".png}
\par\end{centering}

\protect\caption{\label{fig:Testing-of-Check-1}Testing of Check from DB feature}
\end{figure}


Based on Figure \ref{fig:Testing-of-Check-1} it can be concluded
that the application can check files from DB and shows the content
of both file as well as the result.


\subsection{\label{sub:Search-File-Testing}Search File Testing}

This test aims to determine whether the application is able to search
files stored in a database and table. Figure \ref{fig:Testing-of-Search}
shows that the application can find word2.java file that has been
inputted in the text field provided.

\begin{figure}[!tph]
\begin{centering}
\includegraphics[scale=0.5]{\string"gambar ika/uji/ujiSearch\string".png}
\par\end{centering}

\protect\caption{\label{fig:Testing-of-Search}Testing of Search File }
\end{figure}


The conclusion is the application able to search files that have been
stored in a database and show this on the table as an active row,
in other words the search feature goes well.


\subsection{\label{sub:Delete-File-Testing}Delete File Testing}

This stage aims to test whether the application can delete the file
selected in table by the user from the database and table. The scenario
in this test is to delete word2.java file, in Figure \ref{fig:Testing-of-Delete}
shows that the file word2.java successfully removed from the table
and in Figure \ref{fig:word2.java-have-Deleted} that word2.java file
has been deleted from the database.

\begin{figure}[!tph]
\begin{centering}
\includegraphics[scale=0.5]{\string"gambar ika/uji/ujiDelete1\string".png}
\par\end{centering}

\protect\caption{\label{fig:Testing-of-Delete}Testing of Delete File}
\end{figure}


\begin{figure}[!tph]
\begin{centering}
\includegraphics[scale=0.7]{\string"gambar ika/uji/ujiDelete2\string".png}
\par\end{centering}

\protect\caption{\label{fig:word2.java-have-Deleted}word2.java have Deleted from Database}
\end{figure}


Based on Figure \ref{fig:Testing-of-Delete} and Figure \ref{fig:word2.java-have-Deleted}
concluded that the delete feature in this application runs fine.


\subsection{\label{sub:Other-Features-Testing}Other Features Testing}

This test aims to determine whether the About and Help buttons work
well. In Figure \ref{fig:About-Button-Testing} showing JOptionPane
when About button executed, Figure \ref{fig:Help-Button-Testing}
displays when the Help button is run and Figure \ref{fig:Two-Minutes-Tutorial-Video}
displays Two-Minutes Tutorial Video found in Help Button can played.

\begin{figure}[!tph]
\begin{centering}
\includegraphics[scale=0.5]{\string"gambar ika/uji/ujiABOUT\string".png}
\par\end{centering}

\protect\caption{\label{fig:About-Button-Testing}About Button Testing}
\end{figure}


\begin{figure}[!tph]
\begin{centering}
\includegraphics[scale=0.5]{\string"gambar ika/uji/ujiHELP\string".png}
\par\end{centering}

\protect\caption{\label{fig:Help-Button-Testing}Help Button Testing}
\end{figure}


\begin{figure}[!tph]
\begin{centering}
\includegraphics[scale=0.5]{\string"gambar ika/uji/ujiVIDEO\string".png}
\par\end{centering}

\protect\caption{\label{fig:Two-Minutes-Tutorial-Video}Two-Minutes Tutorial Video
Testing}
\end{figure}


Based on Figure \ref{fig:About-Button-Testing}, Figure \ref{fig:Help-Button-Testing}and
Figure \ref{fig:Two-Minutes-Tutorial-Video}, it can be concluded
that the About button, Help button and video tutorial can be run properly.


\section{\label{sec:Functionality-Testing} Functionality Testing}

The results of Dj application testing described in Table \ref{tab:Testing-Results}is
testing of the features that are available and functional requirements
of this application.

\begin{table}[!tph]
\protect\caption{\label{tab:Testing-Results}Testing Results}


\centering{}%
\begin{tabular*}{15cm}{@{\extracolsep{\fill}}|c|>{\raggedright}m{2cm}|>{\raggedright}m{4cm}|>{\raggedright}m{4cm}|>{\centering}p{1.5cm}|}
\hline 
No  & Test Case  & \centering{}Executed Procedures & \centering{}Expected Results  & Result\tabularnewline
\hline 
\hline 
1. & \centering{}Check File from PC & \centering{}User select the source code and target code from PC then
input the threshold value and click the Process button.  & \centering{}The application provides the percentage of the result
and the statement whether the target code is plagiarism or not.  & Succeed\tabularnewline
\hline 
2. & \centering{}Check File from DB & \centering{}User select the source code and target code from the database
then input the threshold and click the Process button. & \centering{}The application provides the percentage of the result
and the statement whether the target code is plagiarism or not. & Succeed\tabularnewline
\hline 
3. & \centering{}Input Threshold & \centering{}When check files from PC or check files from DB, required
threshold value. & \centering{}Users can input the threshold value in the text field
provided & Succeed\tabularnewline
\hline 
4. & \centering{}Save file to Database  & \centering{}User try to check files from PC and click the Process
button. & \centering{}Source code and target code will be saved to the database
automatically. & Succeed\tabularnewline
\hline 
5 & \centering{}Show Content of File & \centering{}When user check files from PC or DB and click the Process
button. & \centering{}Content of source code and target code is displayed on
the text area provided. & Succeed\tabularnewline
\hline 
6. & \centering{}Search File & \centering{}Search File User input the name or size of the file in
text field then click the Search button. & \centering{}The application shows the file is searched in the table. & Succeed\tabularnewline
\hline 
7. & \centering{}Delete File & \centering{}User choose the file to be deleted in the table and then
click the Delete Button. & \centering{}File deleted from the database and tables. & Succeed\tabularnewline
\hline 
8. & \centering{}Watch Tutorial Video & \centering{}User clicks a Two-Minutes Tutorial Video button in Help
Menu.  & \centering{}Two-Minutes Tutorial Video played & Succeed\tabularnewline
\hline 
\end{tabular*}
\end{table}



\section{\label{sec:User-Acceptance-Testing}User Acceptance Testing}

User acceptance testing is perform by implementing the Likert scale
in questionnaire completed by 10 respondents with the given question
is about learnability, satisfaction and effectiveness of Dj application
based on Consolidated and Normative Usability. \citep{Abran} or Table
\ref{tab:Data-questionnaire-respondents} shows the questionnaire
result. 

\begin{table}[!tph]
\protect\caption{\label{tab:Data-questionnaire-respondents}Data questionnaire respondents}


\centering{}%
\begin{tabular}{|c|c|c|c|c|c|}
\hline 
\multirow{2}{*}{Question} & \multicolumn{5}{c|}{Data}\tabularnewline
\cline{2-6} 
 & VA  & A  & N & D  & VD\tabularnewline
\hline 
1. & 2  & 7  & 1  & 0 & 0\tabularnewline
\hline 
2. & 7 & 3 & 0 & 0 & 0\tabularnewline
\hline 
3. & 1 & 8 & 1 & 0 & 0\tabularnewline
\hline 
4. & 2 & 7 & 1 & 0 & 0\tabularnewline
\hline 
5. & 3 & 7 & 0 & 0 & 0\tabularnewline
\hline 
\end{tabular}
\end{table}


After the data has been obtained, the data will be processed with
the following analysis: Categories: 
\begin{enumerate}
\item VA = Very Agree, 5 points 
\item A = Agree, 4 points 
\item N = Neutral, 3 points 
\item D = Disagree, 2 points 
\item VD = Very Disagree, 1 point 
\end{enumerate}
\qquad{}Criterion score = highest score x number of questions x number
of respondents = 5 x 5 x 10 = 250 

Lowest score = lowest score x number of questions x number of respondents
= 1 x 5 x 10 =50. 

Total scores on data collection = 212. 

Total class interval = (Criterion score - Lowest score) / number of
categories = (250-50)/5 = 40 

\begin{table}[!tph]


\protect\caption{\label{tab:Classification-questionnaire-sco}Classification questionnaire
scores expediency}


\begin{centering}
\begin{tabular}{|c|c|c|}
\hline 
No & Total Scores  & Categories\tabularnewline
\hline 
\hline 
1. & 210 -250  & Very Agree \tabularnewline
\hline 
2. & 170-209  & Agree \tabularnewline
\hline 
3. & 130-169  & Neutral \tabularnewline
\hline 
4. & 90-129 & Disagree \tabularnewline
\hline 
5. &  50-89  & Very Disagree \tabularnewline
\hline 
\end{tabular}
\par\end{centering}

\end{table}


The conclusion that can be drawn through the data processing and Table
\ref{tab:Classification-questionnaire-sco}, the total score obtained
by the respondent is included in Agree classification because it is
located between 210 -250.


\chapter{Conclusion}


\section{\label{sec:Conclusion}Conclusion}

From this study, it can be seen that the Dj has successfully implement
the Levenshtein Distance algorithm in the application for detecting
the similarity of the source code. It has developed using the Java
language with the help of software Netbeans. A video tutorial and
tooltip feature have help user in understanding Dj with easily. 

Based on the user acceptance testing showed that the success rate
of Dj application is 84.8\%. It can be concluded that the Dj suitable
for use in detecting plagiarism in Java source code. A weakness of
Dj is a video tutorial can only be played once each time Dj runs. 


\section{\label{sec:Suggestion-and-Future}Suggestion and Future Work}

An algorithm that used in this application still can be optimized
in order to provide better result. Implement another algorithm like
Damerau-Levenshtein Distance which is a development of Levenshtein
distance, design an application works online, improve the existing
GUI and the preprocessing stage can become material research for future
work to get a better result.
\begin{lyxcode}

\end{lyxcode}
\begin{onehalfspace}
\bibliographystyle{apalike}
\nocite{*}
\bibliography{\string"Ika Pretty S\string"}

\end{onehalfspace}

\clearpage{}

\appendix
\begin{appendices}

\renewcommand{\thechapter}{}
\chapter{Chess Class Implementation}

\begin{lstlisting}
#include "Cell.h"
#include "Chess.h"
#include "ChessPiece.h"

#include <vector>
#include <iostream>
#include <stdio.h>

using namespace std;

//konstruktor
Chess::Chess(){
   Chess::row = 8;
   Chess::column = 8;
   Chess::loadChess();
}

//buat load chessboard
void Chess::loadChess(){
   int i, j, k = 0;
   ChessPiece chesspiece;
   for (i=0; i < row; i++){
      for (j=0; j < column; j++){
         //baris 0 dan 1
         if (i<2) {
            if (i==0){
               if (j==0 || j==7){
                  chesspiece.piece='R';
               } else if (j==1 || j==6){
                  chesspiece.piece='N';
               } else if (j==2 || j==5){
                  chesspiece.piece='B';
               } else if (j==3){
                  chesspiece.piece='Q';
               } else {
                  chesspiece.piece='K';
               }
            } else {
               chesspiece.piece='P';
            }
            chesspiece.color=0;
         //baris 2 sampai baris 5
         } else if (i<6) {
            chesspiece.piece='X';
            chesspiece.color=9;
         //baris 6 dan 7
         } else {
            if (i==6){
               chesspiece.piece='P';
            } else {
               if (j==0 || j==7){
                  chesspiece.piece='R';
               } else if (j==1 || j==6){
                  chesspiece.piece='N';
               } else if (j==2 || j==5){
                  chesspiece.piece='B';
               } else if (j==3){
                  chesspiece.piece='Q';
               } else {
                  chesspiece.piece='K';
               }
            }
            chesspiece.color=1;
         }
         chesscell[i][j]=chesspiece;
      }
   }
}

//konversi posisi dari 0,0 jadi a,1
string Chess::Position(Cell cell){
   string position="";
   char column;
   char row;
   switch (cell.column){
   case 0:
      column = 'a';
      break;
   case 1:
      column = 'b';
      break;
   case 2:
      column = 'c';
      break;
   case 3:
      column = 'd';
      break;
   case 4:
      column = 'e';
      break;
   case 5:
      column = 'f';
      break;
   case 6:
      column = 'g';
      break;
   case 7:
      column = 'h';
      break;
   default:
      cout << "bad column";
   }
   switch (cell.row){
   case 0:
      row = '8';
      break;
   case 1:
      row = '7';
      break;
   case 2:
      row = '6';
      break;
   case 3:
      row = '5';
      break;
   case 4:
      row = '4';
      break;
   case 5:
      row = '3';
      break;
   case 6:
      row = '2';
      break;
   case 7:
      row = '1';
      break;
   default:
      cout << "bad column";
   }
   position.push_back(column);
   position.push_back(row);
   return position;
}

//buat replace cell value
void Chess::setChessCell(Cell cell, ChessPiece cp){
   Chess::chesscell[cell.row][cell.column] = cp;
}

//buat replace cell value
void Chess::setChessCell(int row, int column, ChessPiece cp){
   Chess::chesscell[row][column] = cp;
}

//buat ambil value dari sebuah cell
ChessPiece Chess::getChessCellValue(Cell cell){
   return chesscell[cell.row][cell.column];
}

//buat ambil value dari sebuah cell
ChessPiece Chess::getChessCellValue(int row, int column){
   return chesscell[row][column];
}

//cetak chessboard
void Chess::printChess(){
   for (int i=0; i<8; i++){
      for (int j=0; j<8; j++){
         ChessPiece cp = Chess::getChessCellValue(i,j);
         printf("[%c|%d] ", cp.piece, cp.color);
      }
      cout << "\n";
   }
}

//set PGN in Short Algebraic Notation
void Chess::setPGNotation(char piece, string mode1, string position, string mode2){
   pgnotation += piece+mode1+position+mode2+" ";
}

//set PGN in Long Algebraic Notation type 1
void Chess::setLongPGNotation(char piece, string mode1, string positiondeparture, string positiondestination, string mode2){
   pgnotation += piece+positiondeparture+mode1+positiondestination+mode2+" ";
}

//set PGN in Long Algebraic Notation type 2
void Chess::setLongPGNotation(string castling, string mode2){
   pgnotation = castling+mode2+" ";
}

// get PGNotation
string Chess::getPGNotation(){
   return pgnotation;
}

//hapus PGNotation
void Chess::clearPGNotation(){
   pgnotation = "";
}

//menggerakkan chesspieces
bool Chess::moveChessPiece(vector<Cell> movementcells, int turn){
   if (movementcells.size()==4){
      if (Chess::castlingMove(movementcells, turn)){
         return true;
      } else {
         return false;
      }
   } else if (movementcells.size()==3){
      if (Chess::enpassantMove(movementcells, turn)){
         return true;
      } else {
         return false;
      }
   } else if (movementcells.size()==2){
      Chess::normalMove(movementcells, turn);
      return true;
   } else {
      return false;
   }
}

bool Chess::enpassantMove(vector<Cell> movementcells, int turn){
   ChessPiece blankChessPiece;
   ChessPiece cp1, cp2, cp3;
   Cell cell1, cell2, cell3;
   blankChessPiece.piece='X';
   blankChessPiece.color=9;
   string mode1, mode2;
   
   bool enpassant = false;

   //misahin cell sejajar horizontal dan yang lainnya
   //yang sejajar horizontal adalah asal dari pergerakan
   if (movementcells[0].row == movementcells[1].row){
      cell1 = movementcells[0];
      cell2 = movementcells[1];
      cell3 = movementcells[2];
   } else if (movementcells[0].row == movementcells[2].row){
      cell1 = movementcells[0];
      cell2 = movementcells[2];
      cell3 = movementcells[1];
   } else {
      cell1 = movementcells[1];
      cell2 = movementcells[2];
      cell3 = movementcells[0];
   }

   cp1 = Chess::getChessCellValue(cell1);
   cp2 = Chess::getChessCellValue(cell2);
   cp3 = Chess::getChessCellValue(cell3);
   if ( cp1.piece =='P' && cp2.piece =='P' && cp3.piece =='X' ) {
      enpassant = true;
   }

   //cek enpassant
   if(enpassant){

      cout << "\nEn Passant Process..... En Passant Formation" << endl;
      //kalau warna isi dari cell sama dengan turn (giliran) maka departurenya cell1 (cp1)
      //beda dengan normal moves, ini destinasinya ke cell3
      if(cp1.color == (turn%2)){

         //en-passant hanya boleh dilakukan di rank ke 5
         //berarti kalo putih departurnya harus dari row 5
         //kalo hitam departurenya harus dari row 4
         if( (cp1.color == 1 && cell1.row == 3) || (cp1.color == 0 && cell1.row == 4)){
            cout << "En Passant Process..... Departure Cell in proper row\n" << endl;
            printf("cell departure\t\t= [%d,%d]\t\t= [%c|%d]\ncell destinasi\t\t= [%d,%d]\t\t= [%c|%d]\ncell yang dimakan\t= [%d,%d]\t\t= [%c|%d]\n",
                  cell1.row, cell1.column, cp1.piece, cp1.color,
                  cell3.row, cell3.column, cp3.piece, cp3.color,
                  cell2.row, cell2.column, cp2.piece, cp2.color);

            //en-passant memang gerakan makan
            mode1='x';
            //cell3 merupakan destination, maka harus ditiban isinya dengan isi dari cell departure (cell1/cp1)
            Chess::setChessCell(cell3, cp1);
            //karena cp di cell1 udah pindah, maka isinya jadi kosong
            Chess::setChessCell(cell1, blankChessPiece);
            //karena cp di cell2 udah dimakan, maka isinya jadi kosong
            Chess::setChessCell(cell2, blankChessPiece);

            //cek skak dari cell destination
            if (Chess::Check(cell3)) {
               cout << "(skak) ";
               mode2="+";
            }
      
            Chess::setLongPGNotation(cp1.piece, mode1, Chess::Position(cell1), Chess::Position(cell3), "e.p."+mode2);
            return true;
         } else {
            //belum tentu, siapa tau kebalik susunannya
            //do nothing
         }

      //kalau ga terpenuhi maka departurenya cell2 (cp2)
      //beda dengan normal moves, ini destinasinya ke cell3
      } else {
         //en-passant hanya boleh dilakukan di rank ke 5
         //berarti kalo putih departurnya harus dari row 5
         //kalo hitam departurenya harus dari row 4
         if( (cp1.color == 1 && cell1.row == 3) || (cp1.color == 0 && cell1.row == 4)){
            cout << "En Passant Process..... Departure Cell in proper row\n" << endl;
            printf("cell departure\t\t= [%d,%d]\t\t= [%c|%d]\ncell destinasi\t\t= [%d,%d]\t\t= [%c|%d]\ncell yang dimakan\t= [%d,%d]\t\t= [%c|%d]\n",
                     cell2.row, cell2.column, cp2.piece, cp2.color,
                     cell3.row, cell3.column, cp3.piece, cp3.color,
                     cell1.row, cell1.column, cp1.piece, cp1.color);

            //en-passant memang gerakan makan
            mode1='x';
            //cell3 merupakan destination, maka harus ditiban isinya dengan isi dari cell departure (cell2/cp2)
            Chess::setChessCell(cell3, cp2);
            //karena cp di cell2 udah pindah, maka isinya jadi kosong
            Chess::setChessCell(cell2, blankChessPiece);
            //karena cp di cell1 udah dimakan, maka isinya jadi kosong
            Chess::setChessCell(cell1, blankChessPiece);

            //cek skak dari cell destination
            if (Chess::Check(cell3)) {
               cout << "(skak) ";
               mode2="+";
            }
      
            Chess::setLongPGNotation(cp2.piece, mode1, Chess::Position(cell2), Chess::Position(cell3), "e.p."+mode2);
            return true;
         } else {
            return false;
         }
      }

   //kalo formasi cell bukan formasi en-passant
   //berarti bukan en-passant movement
   } else {
      return false;
   }
}

//menggerakkan castling move
bool Chess::castlingMove(vector<Cell> movementcells, int turn){
   ChessPiece cp1, cp2, cp3;
   ChessPiece blankChessPiece;
   Cell destinationking, destinationrook;
   Cell departureking, departurerook;
   blankChessPiece.piece='X';
   blankChessPiece.color=9;
   string mode2;
   bool onerow = true;

   for (int i=0; i<movementcells.size(); i++){
      if ((turn%2)==0) {
         if (movementcells[i].row!=0) {
            onerow=false;
            break;
         }
      } else {
         if (movementcells[i].row!=7) {
            onerow=false;
            break;
         }
      }
   }

   //kalau 1 baris
   if(onerow){
      cout << "\nCastling Process..... All in one row 0 or 7" << endl;
      //cari dimana king
      bool kingcastling = false;
      for (int i=0; i<movementcells.size(); i++){
         cp1 = Chess::getChessCellValue(movementcells[i].row, movementcells[i].column);

         //kalo king
         if (cp1.piece=='K'){
            cout << "Castling Process..... King found" << endl;
            //kalo di column 4
            if(movementcells[i].column==4){
               cout << "Castling Process..... King in the proper cell" << endl;
               //berarti king bisa castling
               kingcastling = true;
               //catat asal king ada dimana
               departureking = movementcells[i];
               break;
            }
         }
      }

      //kalau king bisa castling
      if(kingcastling){
         //cari dimana rook
         int rookcastling = 0;
         for (int i=0; i<movementcells.size(); i++){
            cp2 = Chess::getChessCellValue(movementcells[i].row, movementcells[i].column);
            //kalo rook
            if (cp2.piece=='R'){
               cout << "Castling Process..... Rook found" << endl;

               //kalo di column 0
               if (movementcells[i].column==0){
                  cout << "Castling Process..... Rook in the proper cell" << endl;
                  cout << "Castling Process..... Rook and King can do Queenside Castling" << endl;
                  //berarti queenside castling
                  rookcastling = 1;
                  //catat asal rook ada dimana
                  departurerook = movementcells[i];
                  break;

               } else if (movementcells[i].column==7){
                  cout << "Castling Process..... Rook in the proper cell" << endl;
                  cout << "Castling Process..... Rook and King can do Kingside Castling" << endl;
                  //berarti kingside castling
                  rookcastling = 2;
                  //catat asal rook ada dimana
                  departurerook = movementcells[i];
                  break;

               } else {
                  //berarti salah
                  rookcastling = 0;
                  break;
               }
            }
         }

         //kalau rook bisa queenside castling
         if(rookcastling == 1){
            //cek bener ga yang lainnya empty
            int totalempty = 0;
            for (int i=0; i<movementcells.size(); i++){
               cp3 = Chess::getChessCellValue(movementcells[i].row, movementcells[i].column);

               //kalo kosong
               if (cp3.piece == 'X'){
                  //kalo kosongnya di kolom 2 dan 3 queenside castling dapat dilakukan
                  if (movementcells[i].column==2){
                     //catat tujuan king
                     destinationking = movementcells[i];
                     totalempty++;
                  } else if (movementcells[i].column==3){
                     //catat tujuan rook
                     destinationrook = movementcells[i];
                     totalempty++;
                  }
               }
            }

            //kalo 2 cell lainnya kosong berarti bisa queenside castling
            if (totalempty == 2){
               cout << "Castling Process..... Destination rows are empty" << endl;
               cout << "Castling Process..... Performing Queenside Castling" << endl;

               //saatnya replace2 value
               Chess::setChessCell(destinationking, Chess::getChessCellValue(departureking));
               Chess::setChessCell(destinationrook, Chess::getChessCellValue(departurerook));

               //yang udah pindah di replace blank value
               Chess::setChessCell(departureking, blankChessPiece);
               Chess::setChessCell(departurerook, blankChessPiece);

               if (Chess::Check(destinationrook)) {
                  cout << "(skak) ";
                  mode2="+";
               }
               Chess::setLongPGNotation("O-O-O", mode2);
               return true;
            }

         //kalau rook bisa kingside castling
         } else if (rookcastling == 2) {
            //cek bener ga yang lainnya empty
            int totalempty = 0;
            for (int i=0; i<movementcells.size(); i++){
               cp3 = Chess::getChessCellValue(movementcells[i].row, movementcells[i].column);

               //kalo kosong
               if (cp3.piece == 'X'){
                  //kalo kosongnya di kolom 2 dan 3 queenside castling dapat dilakukan
                  if (movementcells[i].column==5){
                     //catat tujuan rook
                     destinationrook = movementcells[i];
                     totalempty++;
                  } else if (movementcells[i].column==6){
                     //catat tujuan king
                     destinationking = movementcells[i];
                     totalempty++;
                  }
               }
            }

            //kalo 2 cell lainnya kosong berarti bisa queenside castling
            if (totalempty == 2){
               cout << "Castling Process..... Destination rows are empty" << endl;
               cout << "Castling Process..... Performing Kingside Castling" << endl;

               //saatnya replace2 value
               Chess::setChessCell(destinationking, Chess::getChessCellValue(departureking));
               Chess::setChessCell(destinationrook, Chess::getChessCellValue(departurerook));
               //yang udah pindah di replace blank value
               Chess::setChessCell(departureking, blankChessPiece);
               Chess::setChessCell(departurerook, blankChessPiece);

               if (Chess::Check(destinationrook)) {
                  cout << "(skak) ";
                  mode2="+";
               }

               Chess::setLongPGNotation("O-O", mode2);
               return true;
            }

         //kalo rook ga berada ditempatnya atau ada ga ada rook
         //berarti bukan gerakan castling
         } else {
            return false;
         }

      //kalo king ga berada ditempatnya atau ga ada king
      //berarti bukan gerakan castling
      } else {
         return false;
      }
   
   //kalo bukan satu baris gerakannya (di 0 atau 7)
   //berarti bukan gerakan castling
   } else {
      return false;
   }
}

//menggerakan normal move
void Chess::normalMove(vector<Cell> movementcells, int turn){
   Cell cell1 = movementcells[0];
   Cell cell2 = movementcells[1];
   ChessPiece cp1 = Chess::getChessCellValue(cell1);
   ChessPiece cp2 = Chess::getChessCellValue(cell2);
   ChessPiece blankChessPiece;
   string mode1, mode2;
   blankChessPiece.piece='X';
   blankChessPiece.color=9;

   //kalau warna isi dari cell sama dengan turn (giliran) maka cell destination nya cell2, cell departurenya cell1
   if (cp1.color == (turn%2)){

      printf("cell departure\t\t= [%d,%d]\t\t= [%c|%d]\ncell destinasi\t\t= [%d,%d]\t\t= [%c|%d]\n",
         cell1.row, cell1.column, cp1.piece, cp1.color,
         cell2.row, cell2.column, cp2.piece, cp2.color);

      //kalau cell destination nya ada isinya (bukan X), maka tandanya makan;
      if (cp2.piece != 'X'){
         cout << "(makan) ";
         mode1 = "x";
      } else {
         mode1 = "-";
      }

      //cell2 merupakan destination, maka harus ditiban isinya dengan isi dari cell departure
      Chess::setChessCell(cell2, cp1);
      //karena cp di cell1 udah pindah, maka isinya jadi kosong
      Chess::setChessCell(cell1, blankChessPiece);

      if (Chess::Check(cell2)) {
         cout << "(skak) ";
         mode2="+";
      }
      
      Chess::setLongPGNotation(cp1.piece, mode1, Chess::Position(cell1), Chess::Position(cell2), mode2);
      
   //kalau ga terpenuhi maka cell destination nya cell1, cell departurenya cell2
   } else {
      printf("cell departure\t\t= [%d,%d]\t\t= [%c|%d]\ncell destinasi\t\t= [%d,%d]\t\t= [%c|%d]\n",
         cell2.row, cell2.column, cp2.piece, cp2.color,
         cell1.row, cell1.column, cp1.piece, cp1.color);

      //kalau cell destination nya ada isinya (bukan x), maka tandanya makan;
      if (cp1.piece != 'X'){
         cout << "(makan) ";
         mode1="x";
      } else {
         mode1 = "-";
      }

      //cell2 merupakan destination, maka harus ditiban isinya dengan isi dari cell departure
      Chess::setChessCell(cell1, cp2);
      //karena cp di cell1 udah pindah, maka isinya jadi kosong
      Chess::setChessCell(cell2, blankChessPiece);
      if (Chess::Check(cell1)) {
         cout << "(skak) ";
         mode2="+";
      }
      
      Chess::setLongPGNotation(cp2.piece, mode1, Chess::Position(cell2), Chess::Position(cell1), mode2);
   }
}

//bahan buat method check
bool Chess::innerCell(int row, int column){
   if (row < 0 || column > 7){
      return false;
   } else {
      return true;
   }
}

//buat periksa check
bool Chess::Check(Cell cell){
   ChessPiece chesspiece = Chess::getChessCellValue(cell.row, cell.column);
   //ruang skak pion
   if (chesspiece.piece == 'P'){
      if (Chess::innerCell(cell.row-1, cell.column-1) &&
         Chess::getChessCellValue(cell.row-1, cell.column-1).piece=='K' &&
         Chess::getChessCellValue(cell.row-1, cell.column-1).color!=chesspiece.color){
         return true;
      } else if (Chess::innerCell(cell.row-1, cell.column+1) &&
         Chess::getChessCellValue(cell.row-1, cell.column+1).piece=='K' &&
         Chess::getChessCellValue(cell.row-1, cell.column+1).color!=chesspiece.color){
         return true;
      } else if (Chess::innerCell(cell.row+1, cell.column+1) &&
         Chess::getChessCellValue(cell.row+1, cell.column+1).piece=='K' &&
         Chess::getChessCellValue(cell.row+1, cell.column+1).color!=chesspiece.color){
         return true;
      }  else if (Chess::innerCell(cell.row+1, cell.column-1) &&
         Chess::getChessCellValue(cell.row+1, cell.column-1).piece=='K' &&
         Chess::getChessCellValue(cell.row+1, cell.column-1).color!=chesspiece.color){
         return true;
      } else {
         return false;
      }

   //ruang skak benteng
   } else if (chesspiece.piece == 'R'){
      //atas
      int i=1;
      while(Chess::innerCell(cell.row-i, cell.column)){
         if ( Chess::getChessCellValue(cell.row-i, cell.column).piece!='X' ) {
            if( Chess::getChessCellValue(cell.row-i, cell.column).piece=='K' && 
               Chess::getChessCellValue(cell.row-i, cell.column).color!=chesspiece.color){
                  return true;
            } else {
               break;
            }
         }
         i++;
      }

      //kanan
      i=1;
      while(Chess::innerCell(cell.row, cell.column+i)){
         if( Chess::getChessCellValue(cell.row, cell.column+i).piece!='X' ) {
            if( Chess::getChessCellValue(cell.row, cell.column+i).piece=='K' && 
               Chess::getChessCellValue(cell.row, cell.column+i).color!=chesspiece.color){
                  return true;
            } else {
               break;
            }
         }
         i++;
      }

      //bawah
      i=1;
      while(Chess::innerCell(cell.row+i, cell.column)){
         if( Chess::getChessCellValue(cell.row+i, cell.column).piece!='X' ) {
            if( Chess::getChessCellValue(cell.row+i, cell.column).piece=='K' && 
               Chess::getChessCellValue(cell.row+i, cell.column).color!=chesspiece.color){
                  return true;
            } else {
               break;
            }
         }
         i++;
      }

      //kiri
      i=1;
      while(Chess::innerCell(cell.row, cell.column-i)){
         if( Chess::getChessCellValue(cell.row, cell.column-i).piece!='X' ) {
            if( Chess::getChessCellValue(cell.row, cell.column-i).piece=='K' && 
               Chess::getChessCellValue(cell.row, cell.column-i).color!=chesspiece.color){
                  return true;
            } else {
               break;
            }
         }
         i++;
      }
      return false;

   //ruang skak kuda
   } else if (chesspiece.piece == 'N'){
      if (Chess::innerCell(cell.row-2, cell.column+1) &&
         Chess::getChessCellValue(cell.row-2, cell.column+1).piece=='K' &&
         Chess::getChessCellValue(cell.row-2, cell.column+1).color!=chesspiece.color){
         return true;
      } else if (Chess::innerCell(cell.row-1, cell.column+2) &&
         Chess::getChessCellValue(cell.row-1, cell.column+2).piece=='K' &&
         Chess::getChessCellValue(cell.row-1, cell.column+2).color!=chesspiece.color){
         return true;
      } else if (Chess::innerCell(cell.row+1, cell.column+2) &&
         Chess::getChessCellValue(cell.row+1, cell.column+2).piece=='K' &&
         Chess::getChessCellValue(cell.row+1, cell.column+2).color!=chesspiece.color){
         return true;
      } else if (Chess::innerCell(cell.row+2, cell.column+1) &&
         Chess::getChessCellValue(cell.row+2, cell.column+1).piece=='K' &&
         Chess::getChessCellValue(cell.row+2, cell.column+1).color!=chesspiece.color){
         return true;
      } else if (Chess::innerCell(cell.row+2, cell.column-1) &&
         Chess::getChessCellValue(cell.row+2, cell.column-1).piece=='K' &&
         Chess::getChessCellValue(cell.row+2, cell.column-1).color!=chesspiece.color){
         return true;
      } else if (Chess::innerCell(cell.row+1, cell.column-2) &&
         Chess::getChessCellValue(cell.row+1, cell.column-2).piece=='K' &&
         Chess::getChessCellValue(cell.row+1, cell.column-2).color!=chesspiece.color){
         return true;
      } else if (Chess::innerCell(cell.row-1, cell.column+2) &&
         Chess::getChessCellValue(cell.row-1, cell.column-1).piece=='K' &&
         Chess::getChessCellValue(cell.row-1, cell.column-1).piece!=chesspiece.color){
         return true;
      }  else if (Chess::innerCell(cell.row-2, cell.column-1) &&
         Chess::getChessCellValue(cell.row-2, cell.column-1).piece=='K' &&
         Chess::getChessCellValue(cell.row-2, cell.column-1).color!=chesspiece.color){
         return true;
      } else {
         return false;
      }

   } else if (chesspiece.piece == 'B'){
      //serong kanan atas
      int i=1;
      while(Chess::innerCell(cell.row-i, cell.column+i)){
         if( Chess::getChessCellValue(cell.row-i, cell.column+i).piece!='X') {
            if( Chess::getChessCellValue(cell.row-i, cell.column+i).piece=='K' &&
               Chess::getChessCellValue(cell.row-i, cell.column+i).color!=chesspiece.color){
               return true;
            } else {
               break;
            }
         }
         i++;
      }

      //serong kanan bawah
      i=1;
      while(Chess::innerCell(cell.row+i, cell.column+i)){
         if( Chess::getChessCellValue(cell.row+i, cell.column+i).piece!='X' ) {
            if( Chess::getChessCellValue(cell.row+i, cell.column+i).piece=='K' &&
               Chess::getChessCellValue(cell.row+i, cell.column+i).color!=chesspiece.color){
               return true;
            } else {
               break;
            }
         }
         i++;
      }

      //serong kiri bawah
      i=1;
      while(Chess::innerCell(cell.row+i, cell.column-i)){
         if( Chess::getChessCellValue(cell.row+i, cell.column-i).piece!='X') {
            if( Chess::getChessCellValue(cell.row+i, cell.column-i).piece=='K' &&
               Chess::getChessCellValue(cell.row+i, cell.column-i).color!=chesspiece.color){
               return true;
            } else {
               break;
            }
         }
         i++;
      }

      //serong kiri atas
      i=1;
      while(Chess::innerCell(cell.row-i, cell.column-i)){
         if( Chess::getChessCellValue(cell.row-i, cell.column-i).piece!='X') {
            if( Chess::getChessCellValue(cell.row-i, cell.column-i).piece=='K' &&
               Chess::getChessCellValue(cell.row-i, cell.column-i).color!=chesspiece.color){
               return true;
            } else {
               break;
            }
         }
         i++;
      }
      return false;

   } else if (chesspiece.piece == 'Q'){
      //atas
      int i=1;
      while(Chess::innerCell(cell.row-i, cell.column)){
         if ( Chess::getChessCellValue(cell.row-i, cell.column).piece!='X' ) {
            if( Chess::getChessCellValue(cell.row-i, cell.column).piece=='K' && 
               Chess::getChessCellValue(cell.row-i, cell.column).color!=chesspiece.color){
                  return true;
            } else {
               break;
            }
         }
         i++;
      }

      //kanan
      i=1;
      while(Chess::innerCell(cell.row, cell.column+i)){
         if( Chess::getChessCellValue(cell.row, cell.column+i).piece!='X' ) {
            if( Chess::getChessCellValue(cell.row, cell.column+i).piece=='K' && 
               Chess::getChessCellValue(cell.row, cell.column+i).color!=chesspiece.color){
                  return true;
            } else {
               break;
            }
         }
         i++;
      }

      //bawah
      i=1;
      while(Chess::innerCell(cell.row+i, cell.column)){
         if( Chess::getChessCellValue(cell.row+i, cell.column).piece!='X' ) {
            if( Chess::getChessCellValue(cell.row+i, cell.column).piece=='K' && 
               Chess::getChessCellValue(cell.row+i, cell.column).color!=chesspiece.color){
                  return true;
            } else {
               break;
            }
         }
         i++;
      }

      //kiri
      i=1;
      while(Chess::innerCell(cell.row, cell.column-i)){
         if( Chess::getChessCellValue(cell.row, cell.column-i).piece!='X' ) {
            if( Chess::getChessCellValue(cell.row, cell.column-i).piece=='K' && 
               Chess::getChessCellValue(cell.row, cell.column-i).color!=chesspiece.color){
                  return true;
            } else {
               break;
            }
         }
         i++;
      }

      //serong kanan atas
      i=1;
      while(Chess::innerCell(cell.row-i, cell.column+i)){
         if( Chess::getChessCellValue(cell.row-i, cell.column+i).piece!='X') {
            if( Chess::getChessCellValue(cell.row-i, cell.column+i).piece=='K' &&
               Chess::getChessCellValue(cell.row-i, cell.column+i).color!=chesspiece.color){
               return true;
            } else {
               break;
            }
         }
         i++;
      }

      //serong kanan bawah
      i=1;
      while(Chess::innerCell(cell.row+i, cell.column+i)){
         if( Chess::getChessCellValue(cell.row+i, cell.column+i).piece!='X' ) {
            if( Chess::getChessCellValue(cell.row+i, cell.column+i).piece=='K' &&
               Chess::getChessCellValue(cell.row+i, cell.column+i).color!=chesspiece.color){
               return true;
            } else {
               break;
            }
         }
         i++;
      }

      //serong kiri bawah
      i=1;
      while(Chess::innerCell(cell.row+i, cell.column-i)){
         if( Chess::getChessCellValue(cell.row+i, cell.column-i).piece!='X') {
            if( Chess::getChessCellValue(cell.row+i, cell.column-i).piece=='K' &&
               Chess::getChessCellValue(cell.row+i, cell.column-i).color!=chesspiece.color){
               return true;
            } else {
               break;
            }
         }
         i++;
      }

      //serong kiri atas
      i=1;
      while(Chess::innerCell(cell.row-i, cell.column-i)){
         if( Chess::getChessCellValue(cell.row-i, cell.column-i).piece!='X') {
            if( Chess::getChessCellValue(cell.row-i, cell.column-i).piece=='K' &&
               Chess::getChessCellValue(cell.row-i, cell.column-i).color!=chesspiece.color){
               return true;
            } else {
               break;
            }
         }
         i++;
      }
      return false;
   }
}
\end{lstlisting}


\end{appendices}
\end{document}
